% !TeX root = main.tex

\centerline{\sectionfont\the\sectcount.\ Comparison Inequalities}
\medskip
\noindent{\bf\the\sectcount.\the\subsectcount.\ Bochner's Formula.}
In this section, we first introduce a useful tool, namely Bochner's formula.
\medskip
\proclaim Theorem~\propnumber~{\rm(Bochner's formula)}.
Let $(M,g)$ be a Riemannian manifold, $f$ be a smooth function on $M$, then we have
$${1\over2}\Delta|\nabla f|^2=|\Hess f|^2+\langle\nabla\Delta f,\nabla f\rangle+\Ric(\nabla f,\nabla f).$$\par
\noindent{\it Proof.}\/ Since covariant differentiation is clearly commutative with musical isomorphism, we may use musical isomorphism to obtain Bochner's formula for $1$-form:
$${1\over2}\Delta|df|^2=|\nabla df|^2+\langle\nabla f,\nabla\Delta f\rangle+\Ric(\nabla f,\nabla f).$$
Slightly change Ricci identity we obtain
$$\nabla^2_{X,Y}df(Z)-\nabla^2_{Y,X}df(Z)=df(R(X,Y)Z).\eqdef{eq:Bochner 1}$$
Notice that
$$\nabla^2_{X,Y}df(Z)=\nabla_X(\Hess{f(Z,Y)})=\nabla_X(\Hess{f(Y,Z)})=\nabla^2_{X,Z}df(Y),$$
then by contracting $2,3$ indices, we obtain
$$\tr_{2,3}\nabla^2df(Y,\cdot,\cdot)-\nabla_Y\Delta f=\Ric(Y,\Delta f).\eqdef{eq:Bochner 2}$$
Clearly $\nabla_Y\Delta f=\langle Y,\nabla\Delta f\rangle$.
We evaluate the first term.
Consider $\nabla^2_{X,Y}|df|^2$, we have
$$\nabla^2_{X,Y}|df|^2=2\langle\nabla_Xdf,\nabla_Ydf\rangle+2\langle\nabla_X\nabla_Ydf,df\rangle-2\langle\nabla_{\nabla_XY}df,df\rangle.$$
Contracting $X,Y$, we obtain
$${1\over2}\Delta|df|^2=|\nabla df|^2+\tr_{2,3}\nabla^2df(\nabla f,\cdot,\cdot).$$
Take $Y=\nabla f$ in~\eqref{eq:Bochner 2}, we obtain equation~\eqref{eq:Bochner 1}.
\advance\propcount by 1
\qed
\noindent{\it Remark~\propnumber.}\/ For further computation, we notice that $|\Hess f|^2$ can usually be computed by the sum of squares of eigenvalues of $\Hess f$.
\advance\propcount by 1
\advance\subsectcount by 1
\bigskip
\noindent{\bf\the\sectcount.\the\subsectcount.\ Some Computations.}
We are concerned about volume form of geodesic balls and mean curvature of geodesic spheres, they are connected by the Hessian of distance functions.
For this, we do some calculations.
\medskip
\proclaim Lemma~\propnumber. Let $(M,g)$ be a Riemannian manifold, $r$ be a distance function with respective to a point $p$.
Then within cut locus of $p$, $\Hess{r}=\II_{\partial B_p(\rho)}$, with the normal vector field to be $\nabla r$.\par
\noindent{\it Proof.}\/ First we notice that by Gauss lemma, under geodesic polar coordinate, the metric $g$ has form
$$g=dr^2+g_{ij}d\theta^i\otimes d\theta^j,$$
hence $\nabla r$ is indeed a normal vector field.
Let $X,Y\in T_q\partial B_p(\rho)$, we have
$$\eqalign{
	\Hess{r}(X,Y)&=XYr-(\nabla_XY)r\cr
	&=X\langle Y,\nabla r\rangle-\langle\nabla_XY,\nabla r\rangle\cr
	&=X\langle Y,\nabla\rangle-X\langle Y,\nabla r\rangle+\langle Y,\nabla_X\nabla r\rangle\cr
	&=\langle S(X),Y\rangle,
}$$
where $S$ is the shape operator.
Hence we have $\Hess{r}=\II$.
\advance\propcount by 1
\qed
\noindent{\it Remark~\propnumber.}\/ In contrast to some sign convention, we simply ask second fundamental form and shape operator are associated bilinear form and linear transformation, we don't ask they differ a minus sign.
\advance\propcount by 1
\medskip
By taking contraction, we obtain the following
\medskip
\definexref{cor:mean curv laplacian}{\propnumber}{cor}
\proclaim Corollary~\propnumber. We have $\Delta r=m$, where $m$ is the mean curvature of $\partial B_p(\rho)$.\par
\advance\propcount by 1
Let the volume form of $B_p(\rho)$ be
$$d\vol{}={\cal A}(r,\theta)dr\wedge d\theta^1\wedge\cdots\wedge d\theta^{n-1}.\eqdef{eq:volume form}$$
Then we have
\medskip
\definexref{lem:mean curv}{\propnumber}{lem}
\proclaim Lemma~\propnumber. Let ${\cal A}'$ be the derivative of $\cal A$ with respective to $r$, then
$${{\cal A}'(r,\theta)\over{\cal A}(r,\theta)}=m(r).$$\par
\noindent{\it Proof.}\/ Let's just calculate.
We have
$$\eqalign{
	{{\cal A}'(r,\theta)\over{\cal A}(r,\theta)}&={d\over dr}\log{\cal A}(r,\theta)\cr
	&={d\over dr}\log\sqrt{\det(g_{ij})}\cr
	&={1\over2}{1\over\det(g_{ij})}\det(g_{ij})g^{ij}\partial_r\langle\partial_i,\partial_j\rangle\cr
	&=g^{ij}\langle\nabla_r\partial_i,\partial_j\rangle\cr
	&=g^{ij}\langle\nabla_i\nabla r,\partial_j\rangle\cr
	&=m(r).}$$
\advance\propcount by 1
\advance\subsectcount by 1
\qed
\noindent{\bf\the\sectcount.\the\subsectcount.\ Comparison Inequalities.}
In this subsection we introduce comparison inequalities.
The first one is the mean curvature comparison.
We denote $B^H(\rho)$ the geodesic ball of radius $\rho$ in the space form $S^2(H)$ of constant curvature $H$.
\medskip
\definexref{thm:mean curv}{\propnumber}{thm}
\proclaim Theorem~\propnumber~{\rm(Mean curvature comparison)}.
Let $(M,g)$ be a Riemannian manifold with $\Ric\geq(n-1)H$, then along any minimal geodesic segement from $p$, we have
$$m(\rho)\leq m_H(\rho),$$
where $m_H(\rho)$ denotes the mean curvature of $\partial B^H(\rho)$.\par
\noindent{\it Proof.}\/ Plug $f=r$ into Bochner's formula, notice that $|\nabla r|=1$, we obtain
$$0=|\Hess r|^2+\langle\nabla r,\nabla m\rangle+\Ric(\nabla r,\nabla r).$$
By Cauchy--Schwarz inequality, we have
$$|\Hess r|^2=|\II{}|^2\geq{m^2\over n-1}.$$
Moreover, we have
$$\Ric(\nabla r,\nabla r)\geq(n-1)H.$$
Thus we obtain
$$m'=\langle\nabla r,\nabla m\rangle\leq-{m^2\over n-1}-(n-1)H.$$
For space form, the equality holds, that is
$$m'_H=-{m_H^2\over n-1}-(n-1)H.$$
Since $\lim_{r\to 0}(m-m_H)=0$, by standard Riccati equation comparison, we then obtain the result.
\advance\propcount by 1
\qed
The second theorem is auxiliary, but we still call it a theorem.
\medskip
\definexref{thm:Ricci comparison}{\propnumber}{thm}
\proclaim Theorem~\propnumber~{\rm(Ricci comparison)}.
Let $(M,g)$ be a Riemannian manifold with $\Ric\geq(n-1)H$.
Let ${\cal A}(r,\theta)$ be as in~\eqref{eq:volume form}, ${\cal A}_H(r,\theta)$ be similar for space form $S^2(H)$.
Then along any minimal geodesic segement from $p$,
$${\cal A}(r,\theta)\over{\cal A}_H(r,\theta)\eqdef{eq:Ricci comparison}$$
is nonincreasing with respective to $r$.\par
\noindent{\it Proof.}\/ Notice that the logarithm derivative for~\eqref{eq:Ricci comparison}~is
$$\eqalign{
	{d\over dr}\log\left({{\cal A}(r,\theta)\over{\cal A}_H(r,\theta)}\right)&={{\cal A}'(r,\theta)\over{\cal A}(r,\theta)}-{{\cal A}'_H(r,\theta)\over {\cal A}_H(r,\theta)}\cr
	&=m(r)-m_H(r)\cr
	&\leq 0,
}$$
here we used~\ref{lem:mean curv}~and mean curvature comparison inequality.
\advance\propcount by 1
\qed
Our last theorem for this section is Bishop--Gromov volume comparison inequality.
This is the core of our seminar.
\proclaim Theorem~\propnumber~{\rm(Volume comparison)}.
Let $(M,g)$ be a Riemannian manifold with $\Ric\geq(n-1)H$, then
$$\vol(B_p(\rho))\over\vol(B^H(\rho))$$
is nonincreasing with respective to $\rho$.\par
\noindent{\it Proof.}\/ We have
$$\eqalign{
	\vol(B_p(\rho))&=\int_{S^{n-1}}\int_0^\rho{\cal A}(r,\theta)dr\wedge d\theta^1\wedge\cdots\wedge d\theta^{n-1},\cr
	\vol(B^H(\rho))&=\int_{S^{n-1}}\int_0^\rho{\cal A}_H(r,\theta)dr\wedge d\theta^1\wedge\cdots\wedge d\theta^{n-1}.
}$$
Take derivative with respective to $\rho$, we have
$$\eqalign{
	{d\over d\rho}\left({\vol(B_p(\rho))\over\vol(B^H(\rho))}\right)&={\left(\int_{S^{n-1}}{\cal A}(\rho,\theta)d\vol_{S^{n-1}}\right)\left(\int_{S^{n-1}}\int_0^\rho{\cal A}_H(r,\theta)dr\wedge d\vol_{S^{n-1}}\right)\over(\vol(B^H(\rho))^2}\cr
	&\quad-{\left(\int_{S^{n-1}}{\cal A}_H(\rho,\theta)d\vol_{S^{n-1}}\right)\left(\int_{S^{n-1}}\int_0^\rho{\cal A}(r,\theta)dr\wedge d\vol_{S^{n-1}}\right)\over(\vol(B^H(\rho)))^2}\cr
	&=(\vol(B^H(\rho))^{-2}\int_0^\rho\biggl(\left(\int_{S^{n-1}}{\cal A}(\rho,\theta)d\vol_{S^{n-1}}\right)\left(\int_{S^{n-1}}{\cal A}_H(r,\theta)d\vol_{S^{n-1}}\right)\cr
	&\quad-\left(\int_{S^{n-1}}{\cal A}_H(\rho,\theta)d\vol_{S^{n-1}}\right)\left(\int_{S^{n-1}}{\cal A}(r,\theta)d\vol_{S^{n-1}}\right)\biggr)dr.
}$$
Therefore, to check
$$\rho\mapsto{\vol(B_p(\rho))\over\vol(B^H(\rho))}$$
is nonincreasing, it's suffice to check
$$r\mapsto{\int_{S^{n-1}}{\cal A}(r,\theta)d\vol_{S^{n-1}}\over\int_{S^{n-1}}{\cal A}_H(r,\theta)d\vol_{S^{n-1}}}$$
is nonincreasing.
But we have
$${\int_{S^{n-1}}{\cal A}(r,\theta)d\vol_{S^{n-1}}\over\int_{S^{n-1}}{\cal A}_H(r,\theta)d\vol_{S^{n-1}}}={1\over\omega_{n-1}}\int_{S^{n-1}}{{\cal A}(r,\theta)\over{\cal A}_H(r,\theta)}d\vol_{S^{n-1}},$$
which is nonincreasing by~\ref{thm:Ricci comparison}.
\advance\propcount by 1
\advance\subsectcount by 1
\qed
\noindent{\bf\the\sectcount.\the\subsectcount.\ Comparison Inequality in Weak Sense.}
Sometimes we will deal with functions with bad smoothness, so we need comparison theorems in weak sense.
Here we introduce a special case of Laplacian comparison inequality in distribution sense.\par
First, notice by~\ref{cor:mean curv laplacian}, we have
\medskip
\proclaim Proposition~\propnumber.
Under same assumption of~\ref{thm:mean curv}, mean curvature comparison inequality is equivalent to the following\/ {\rm Laplacian comparison inequality}
$$\Delta r\leq\Delta r_H.$$
\advance\propcount by 1
\par
\noindent{\bf Example~\propnumber.}
In our seminar, we are concerned about $H=0$ case the most.
We haven't computed $\Hess r$ for ${\bb R}^n$ yet.
Let's compute for $p=0$ and $\Hess r(\rho)$.
This is just the second fundamental form of sphere
$$(x^1)^2+(x^2)^2+\cdots+(x^n)^2=\rho^2.$$
Fix the outward normal vectot field $N={1\over\rho}{\bf x}$, then we have
$$\II(X,Y)=\langle\nabla_XN,Y\rangle=\left\langle{1\over\rho}X{\bf x},Y\right\rangle={1\over\rho}\langle X,Y\rangle.$$
Hence we have
$$\Hess r={1\over r}(g-dr\otimes dr).$$
Take contraction we have
$$\Delta r={n-1\over r}.$$
\advance\propcount by 1
\medskip
We avoided using Jacobi field to compute Hessian of distance function.
This method is limited, and is not recommended.
\medskip
\proclaim Theorem~\propnumber~{\rm(Laplacian comparison)}~.
Let $(M,g)$ be a Riemannian manifold with $\Ric\geq 0$.
Then for the distance function to $p$, the inequality
$$\Delta r\leq{n-1\over r}$$
holds in distribution sense, that is, for any $\varphi\in C^\infty_0(M),\ \varphi\geq 0$, we have
$$\int_Mr\Delta\varphi\leq\int_M{n-1\over r}\cdot\varphi.$$
\par
\noindent{\it Proof.}\/ Let $M$ be decomposed into cut locus with respective to $p$ and a star-shaped domain, namely $M=:\Omega\sqcup{\rm Cut}(p)$.
Lipschitz function $r$ is differentiable in $\Omega$, thus within $\Omega$ we have
$$\Delta r\leq{n-1\over r}.$$
Fix $\varphi\in C^\infty_0(M),\ \varphi\geq 0$.
Since $|{\rm Cut}(p)|=0$, we have
$$\int_Mr\Delta\varphi=\int_\Omega r\Delta\varphi.$$
Since $\Omega$ is star-shaped, we can choose an increasing sequence $\Omega_k\subset\Omega$ such that
$$\lim_{k\to\infty}\Omega_k=\Omega,$$
and each $\Omega_k$ is obtained by shrinking $\Omega$ along $r$'s direction.
Since Stokes' formula is valid for Lipschitz functions, and $\varphi$ has compact support, we have
$$\eqalign{
    \int_\Omega r\Delta\varphi&=-\int_\Omega\langle\nabla\varphi,\nabla r\rangle\cr
    &=-\lim_{k\to\infty}\int_{\Omega_k}\langle\nabla\varphi,\nabla r\rangle.
}$$
Last equality holds since $|\nabla r|=1$ within $\Omega$ and $\nabla\varphi$ is bounded, then apply Lebesgue's dominated convergence theorem.
By Green's formula, we have
$$-\int_{\Omega_k}\langle\nabla r,\nabla\varphi\rangle=\nabla_{\Omega_k}\Delta r\cdot\varphi-\int_{\partial\Omega_k}\varphi\cdot{\partial r\over\partial\nu},$$
where $\nu$ is the unit outer normal vector field.
Since $\Omega_k$ is obtained by shrinking $\Omega$ along $r$'s direction and $\varphi\geq 0$, we have
$$\int_{\partial\Omega_k}\varphi\cdot{\partial r\over\partial\nu}\geq 0.$$
Thus we have
$$\eqalign{
    -\int_{\Omega_k}\langle\nabla r,\nabla\varphi\rangle&\leq\int_{\Omega_k}\Delta r\cdot\varphi\cr
    &\leq\int_{\Omega_k}{n-1\over r}\cdot\varphi.
}$$
Finally we have
$$\eqalign{
    \int_Mr\Delta\varphi&\leq\lim_{k\to\infty}\int_{\Omega_k}{n-1\over r}\cdot\varphi\cr
    &=\int_\Omega{n-1\over r}\cdot\varphi\cr
    &=\int_M{n-1\over r}\cdot\varphi,
}$$
we used Lebesgue's dominated convergence theorem and $|{\rm Cut}(p)|=0$ again here.
\advance\propcount by 1
\qed
\subsectcount=1
\propcount=1
\eqnumber=0
\advance\sectcount by 1
\bigskip

\section{Rigidity Results}

In this section, we present volume rigidity and diameter rigidity for closed and open manifolds respectively.

\subsection{Volume Rigidity}

Volume rigidity builds on the following rigidity result for volume comparison inequality.
\begin{prop}\label{prop:volume rigidity}
    Let $(M,g)$ be a Riemannian manifold with $\Ric\geq(n-1)H$.
    If there is an $R>0$ such that $\vol(B_p(R))=\vol(B^H(R))$, then $B_p(R)$ must be isometric to $B^H(R)$.
\end{prop}
\begin{proof}
    By volume comparison theorem, the function
    \[r\mapsto\frac{\vol(B_p(r))}{\vol(B^H(r))}\leq 1\]
    is nonincreasing, so if $\vol(B_p(R))=\vol(B^H(R))$ holds for some $R>0$, then $\vol(B_p(\rho))=\vol(B^H(\rho))$ holds for all $0<\rho<R$.
    Clearly we are before the cut locus.
    Let's trace back when the equality holds in volume comparison inequality.
    We used Cauchy--Schwarz inequality in
    \[|\Hess{r}|^2=|\II|^2\geq\frac{m^2}{n-1},\]
    hence if equality holds, $\II$ has equal eigenvalues.
    Thus $\partial B_p(\rho)$ is totally umbilical, in particular, $m(\rho,\theta)$ is constant on $\partial B_p(\rho)$.
    Notice that the metric of $M$ under geodesic polar coordinate is
    \[g=dr\otimes dr+f(r,\theta)g_{S^{n-1}},\]
    take covariant differentiation and we obtain
    \[0=\Hess{r}\otimes dr+dr\otimes\Hess{r}+\nabla f(r,\theta)\otimes g_{S^{n-1}}.\]
    Take contraction we obtain
    \[0=\Delta rdr+\tr_{1,2}dr\otimes\Hess{r}+df(r,\theta),\]
    if we write $\tr_{1,2}dr\otimes\Hess{r}=h(r)dr$, we have
    \[m(r,\theta)dr=-h(r)dr-df(r,\theta).\]
    Since $m$ is constant with respective to $\theta$, we have $f$ is constant with respective to $\theta$.
    Notice that $S^n(H)$ has metric
    \[g_H=dr\otimes dr+\sn^2_H(r)g_{S^{n-1}}.\]
    Plug this into $\vol(B_p(\rho))=\vol(B^H(\rho))$, we have
    \[\int_0^\rho\int_{S^{n-1}}\sqrt{f(r)\det(g_{ij})}d\vol_{S^{n-1}}dr=\int_0^\rho\int_{S^{n-1}}\sqrt{\sn^2_H(r)\det(g_{ij})}d\vol_{S^{n-1}}dr.\]
    Notice that
    \[\int_{S^{n-1}}\sqrt{\det(g_{ij})}d\vol_{S^{n-1}}=\omega_{n-1},\]
    we thus obtain
    \[\int_0^\rho\sqrt{f(r)}dr=\int_0^\rho{\sn_H(r)}dr\]
    holds for all $0<\rho<R$.
    Therefore $f(r)=\sn^2_H(r)$ on $(0,R]$, and $B_p(R)$ is isometric to $B^H(R)$.
\end{proof}

Using Proposition~\ref{prop:volume rigidity}, we can derive volume rigidity for closed and open manifolds.

\begin{thm}[Volume rigidity for closed manifolds]
    Let $(M,g)$ be a Riemannian manifold with $\Ric\geq n-1$.
    If $\vol{M}=\vol{S^n}$, then $M$ is isometric to $S^n$.
\end{thm}
\begin{proof}
    By Bonnet--Myers' theorem, we have $\operatorname{diam}{M}\leq\pi$, hence
    \[\vol{(B_p(\pi))}=\vol{M}=\vol{S^n}=\vol{(B^1(\pi))}.\]
    Then apply Proposition~\ref{prop:volume rigidity}.
\end{proof}

\begin{thm}[Volume rigidity for open manifolds]
    Let $(M,g)$ be a Riemannian manifold with $\Ric\geq 0$.
    If
    \[\lim_{\rho\to\infty}\frac{n\vol{(B_p(\rho))}}{\omega_n\rho^n}=1,\]
    then $M$ is isometric to $\mathbb{R}^n$.
\end{thm}
\begin{proof}
    The function
    \[\rho\mapsto\frac{n\vol{(B_p(\rho))}}{\omega_n\rho^n}=\frac{\vol{(B_p(\rho))}}{\vol{(B^0(\rho))}}\]
    is nonincreasing and has limit $1$, hence
    \[\vol(B_p(\rho))\geq\vol(B^0(\rho)).\]
    By volume comparison inequality, the equality must hold, then apply Proposition~\ref{prop:volume rigidity}~to all $\rho>0$.
\end{proof}

\subsection{Diameter Rigidity}

Our result on diameter rigidity for closed manifolds is Shiu-Yuen Cheng's maximal diameter theorem.

\begin{thm}[Maximal diameter]
    Let $(M,g)$ be a Riemannian manifold with $\Ric\geq n-1$ and $\operatorname{diam}{M}=\pi$, then $M$ is isometric to $S^n$.
\end{thm}
\begin{proof}
    First notice that $M$ is compact by Bonnet--Myers' theorem, so we can choose $p,q\in M$ such that $|pq|=\pi$.
    Consider $B_p(\rho)$ and $B_q(\pi-\rho)$, by triangle inequality, we have $B_p(\rho)\cap B_q(\pi-\rho)=\varnothing$.
    Hence by volume comparison inequality, we have
    \begin{align*}
        \vol{M}&\geq\vol(B_p(\rho))+\vol(B_q(\pi-\rho))\\
        &=\frac{\vol(B_p(\rho))}{\vol(B^1(\rho))}\cdot\vol(B^1(\rho))+\frac{\vol(B_q(\pi-\rho))}{\vol(B^1(\pi-\rho))}\cdot\vol(B^1(\pi-\rho))\\
        &\geq\frac{\vol(B_p(\pi))}{\vol(B^1(\pi))}\cdot\vol(B^1(\rho))+\frac{\vol(B_q(\pi))}{\vol(B^1(\pi))}\cdot\vol(B^1(\pi-\rho))\\
        &=\frac{\vol{M}}{\vol{S^n}}\cdot(\vol(B^1(\rho))+\vol(B^1(\pi-\rho)))\\
        &=\frac{\vol{M}}{\vol{S^n}}\cdot\vol{S^n}\\
        &=\vol{M}.
    \end{align*}
    Therefore the equality of volume comparison inequality holds for $0<\rho\leq\pi$ on $M$, applying Proposition~\ref{prop:volume rigidity}~we obtain $B_p(\rho)$ is isometric to $B^1(\rho)$.
    In particular, $M=B_p(\pi)$ is isometric to $B^1(\pi)=S^n$.
\end{proof}

We need to explain what does ``diameter rigidity'' mean for an open manifold.
Being of infinite diameter is not enough, there should be two points at infinity.
That is, there should be a geodesic adjoining two points at infinity, namely, a line exists.

\begin{defn}
    Let $\gamma:\mathbb{R}\to M$ be a geodesic, if $\gamma$ is a global isometry, i.\ e.\ for any $s,t\in\mathbb{R}$ the equality $|\gamma(s)\gamma(t)|=|s-t|$ holds, then $\gamma$ is called a \emph{line}.
\end{defn}

We expect rigidity result happens when a line exists.
This leads to Cheeger--Gromoll's splitting theorem.

\begin{thm}[Splitting]
    Let $(M,g)$ be a Riemannian manifold with $\Ric\geq 0$, and a line exists on $M$.
    Then $M$ is isometric to $\mathbb{R}\times N$ with product metric.
\end{thm}

Before proving this theorem, we need to input several big theorems.

\begin{thm}[Strong maximum principle]
    Let $(M,g)$ be a Riemannian manifold, $f$ be a Lipschitz function on $M$.
    If $f$ satisfies $\Delta f\leq 0$ in the distribution sense and has a global minimum, then $f$ is constant on $M$.
\end{thm}

\begin{thm}[Weyl's lemma]
    Let $(M,g)$ be a Riemannian manifold, $f$ be a Lipschitz function.
    If $\Delta f=0$ in the distribution sense, then $f$ is smooth.
\end{thm}

\begin{thm}[De Rham decomposition]
    Let $(M,g)$ be a Riemannian manifold.
    Let $\mathcal{F}$ be a distribution that is parallel with respective to the Levi--Civita connection, $\mathcal{F}^\perp$ be its orthogonal complement.
    For any $p\in M$, let $N,N^\perp$ be the maximal integral submanifold of $\mathcal{F},\mathcal{F}^\perp$ at $p$.
    Equip $N,N^\perp$ with induced metric from $M$, then $M$ is isometric to $N\times N^\perp$ with product metric.
\end{thm}

We will not prove these three theorems here.

\begin{proof}[Proof of splitting theorem]
    The proof uses \emph{Busemann function}.
    We first define and discuss some properties of Busemann function.

    Fix $p\in M$, let $\gamma$ be a line on $M$.
    For $t\in\mathbb{R}$, define $b^+_t(p)=|p\gamma(t)|-t$.
    Then by triangle inequality, if $t_1>t_2$, we have
    \[b^+_{t_1}(p)-b^+_{t_2}(p)=|p\gamma(t_1)|-|p\gamma(t_2)|-(t_1-t_2)\leq 0,\]
    hence $b^+_t(p)$ is nonincreasing with respective to $t$.
    Moreover, by triangle inequality, we have $b^+_t(p)\geq-|p\gamma(0)|$, hence $b^+_t(p)$ is bounded below.
    Thus we can take limit
    \[B^+(p):=\lim_{t\to+\infty}b^+_t(p).\]
    Similarly we define $b^-_t(p)=|p\gamma(-t)|-t$, we have the limit
    \[B^-(p):=\lim_{t\to+\infty}b^-_t(p)\]
    is well-defined.

    Now we discuss the properties of Busemann functions.
    First we notice that
    \[B^+(p)+B^-(p)=|p\gamma(t)|+|p\gamma(-t)|-2t\geq 0.\]
    Then by Laplacian comparison inequality, we have for any test function $\varphi\in C^\infty_0(M)$, $\varphi\geq 0$
    \[\int_{M}b^+_t\Delta\varphi\leq\int_M\frac{n-1}{b^+_t+t}\cdot\varphi.\]
    Let $t\to+\infty$, we obtain (we may use Lebesgue's dominated convergence theorem)
    \[\int_MB^+\Delta\varphi\leq 0.\]
    Similarly we have
    \[\int_MB^-\Delta\varphi\leq 0.\]
    Thus $\Delta(B^++B^-)\leq 0$ in the distribution sense.
    Moreover, clearly $B^++B^-=0$ on $\gamma$, hence $B^++B^-$ satisfies the assumption of strong maximum principle, we have $B^++B^-=0$.
    In particular, we have $\Delta(B^++B^-)=0$ in the classical sense, hence $\Delta B^+=\Delta B^-=0$ in the distribution sense.
    By Weyl's lemma, we have $B^+$ and $B^-$ are both smooth.
    Moreover, since the gradient of distance function has modulus $1$ almost everywhere, we have $|\nabla B^+|=1$.
    Using Bochner's formular, we have
    \[0=|\Hess{B^+}|^2+\langle\nabla B^+,\nabla\Delta B^+\rangle+\Ric(\nabla B^+,\nabla B^+)\geq|\Hess{B^+}|^2,\]
    hence $\Hess{B^+}=0$.

    The rest of part is easy geometry.
    Since covariant differentiation is commutative with musical isomorphism, we have $\nabla(\nabla B^+)=(\Hess{B^+})^\sharp=0$.
    Hence $\nabla B^+$ is parallel.
    Choose a regular point $p$ of $B^+$, by de Rham's decomposition theorem, $M$ is isometric to the product of integral curve of $\nabla B^+$ through $p$ and the level set of $B^+$ at $p$.
    Without loss of generality we can assume $p$ lies on $\gamma$, then since $\nabla B^+$ is parallel, the integral curve is the geodesic through $p$, i.\ e.\ the line.
\end{proof}
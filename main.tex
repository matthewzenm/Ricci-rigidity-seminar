\pdfpagewidth=210mm
\pdfpageheight=297mm
% \hoffset=25mm
% \voffset=25mm
\hsize=160mm  % 210 - 2*25
% \vsize=247mm  % 297 - 2*25

\input eplain
\beginpackages
\usepackage{color}
\usepackage{url}
\endpackages
\enablehyperlinks
\hlopts{bwidth=0}

\font\titlefont=txb at 14pt
\font\authorfont=txr at 12pt
\font\sectionfont=txsc at 12pt
\font\footnotefont=txr at 7pt
\input fontsetup

\newcount\sectcount
\newcount\subsectcount
\newcount\propcount
\def\propnumber{{\the\sectcount.\the\propcount}}
\def\eqconstruct#1{\the\sectcount.#1}
\def\sectionword{Section}
\def\thmword{Theorem}
\def\axword{Axiom}
\def\propword{Proposition}
\def\defnword{Definition}
\def\lemword{Lemma}
\def\corword{Corollary}
\def\remword{Remark}

\def\qed{\nobreak\hfill\hbox{\vrule height 6pt width 6pt depth 0pt}\medskip}
\def\tr{\mathop{\rm tr}\nolimits}
\def\Hess{\mathop{\rm Hess}\nolimits}
\def\Ric{\mathop{\rm Ric}\nolimits}
\def\II{\mathop{\rm II}\nolimits}
\def\vol{\mathop{\rm vol}\nolimits}
\def\sn{\mathop{\rm sn}\nolimits}
\def\diam{\mathop{\rm diam}\nolimits}

\centerline{\titlefont Notes on Rigidity Results for Manifolds with Lower}
\smallskip
\centerline{\titlefont Ricci Curvature Bound}
\medskip
\centerline{\authorfont Mengchen Zeng\raise 1ex \hbox{\footnotefont\dag}}\vfootnote\dag{School of Mathematical Sciences, Beijing Normal University.}
\medskip
\centerline{\authorfont\today}
\bigskip

In this seminar, all Riemannian manifolds are assumed to be complete.
The manifold $M$ is understood to be $n$-dimensional if not specially mentioned.
\bigskip
\sectcount=1
\subsectcount=1
\propcount=1
\eqnumber=0

\input tensor
\input comparison
\input rigidity

\bye

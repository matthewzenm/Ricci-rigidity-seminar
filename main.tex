\documentclass{amsart}

\usepackage{amsmath}
\usepackage{amssymb}
\usepackage{amsthm}
\usepackage[shortlabels]{enumitem}
\usepackage[colorlinks]{hyperref}

\theoremstyle{definition}
\newtheorem{defn}{Definition}[section]
\newtheorem{eg}[defn]{Example}
\theoremstyle{plain}
\newtheorem{thm}[defn]{Theorem}
\newtheorem{prop}[defn]{Proposition}
\newtheorem{lem}[defn]{Lemma}
\newtheorem{cor}[defn]{Corollary}
\theoremstyle{remark}
\newtheorem{rem}[defn]{Remark}

\numberwithin{equation}{section}

\newcommand{\R}[4]{R_{#1}{}_{#2}{}_{#3}{}^{#4}}
\DeclareMathOperator{\vol}{vol}
\DeclareMathOperator{\Sect}{Sect}
\DeclareMathOperator{\Ric}{Ric}
\DeclareMathOperator{\Scal}{Scal}
\DeclareMathOperator{\tr}{tr}
\DeclareMathOperator{\Hess}{Hess}
\DeclareMathOperator{\II}{II}
\DeclareMathOperator{\Cut}{Cut}

\title[Ricci Rigidity Notes]{Notes on Rigidity Results for Manifolds with Lower Ricci Curvature Bound}
\author{Mengchen Zeng}
\date{\today}
\address{School of Mathematical Sciences, Beijing Normal University, Beijing, China, P.\ R., 100875.}

\begin{document}

\maketitle

In this seminar, all Riemannian manifolds are assumed to be complete.

\section{Tensors}

\subsection{Notation}

We first fix the notation of tensors.
Let $(M,g)$ be a Riemannian manifold.
A \emph{tensor} $T$ of type $(r,s)$ is a smooth section of vector bundle
\[T^{(r,s)}(TM)=\left(\bigotimes_{i=1}^rTM\right)\otimes\left(\bigotimes_{j=1}^sT^*M\right).\]
On each fiber, a tensor can be regarded as a multilinear map
\[T|_{p}:\underbrace{T^*_pM\times\cdots\times T^*_pM}_{r}\times\underbrace{T_pM\times\cdots\times T_pM}_{s}\to\mathbb{R},\]
so we can talk about symmetric and positive-definite tensor (for $(0,2)$-tensor), alternating tensor (for $(0,n)$-tensor) well.
Tensors of type $(r,0)$ are called contravariant tensors, and of type $(0,s)$ are called covariant tensors.

Let $(x^1,\cdots,x^n)$ be a local coordinate.
We will adopt the notation $\{\partial_1,\cdots,\partial_n\}$ for a local frame for $TM$, and $\{dx^1,\cdots,dx^n\}$ for a local frame for $T^*M$.
We will adopt the \textsc{Einstein} summation convention, so the local expression for an $(r,s)$-tensor $T$ is
\[T=T^{i_1\cdots i_r}_{j_1\cdots j_s}\partial_{i_1}\otimes\cdots\otimes\partial_{i_r}\otimes dx^{j_1}\otimes\cdots\otimes dx^{j_s}.\]

The tangent space $TM$ and cotangent space $T^*M$ are canonically isomorphic via \emph{musical isomorphism}.
Under a local coordinate $(x^1,\cdots,x^n)$, let $g=g_{ij}$.
Then we have the musical isomorphism (``lowering index'')
\begin{align*}
    \flat:TM&\to T^*M\\
    X^i\partial_i&\mapsto g_{jk}X^kdx^j,
\end{align*}
and we denote $X=X^i\partial_i$, $X^\flat=g_{jk}X^kdx^j$.
The inverse is given by (``raising index'')
\begin{align*}
    \sharp:T^*M&\to TM\\
    \omega_idx^i&\mapsto g^{jk}\omega_k\partial_j,
\end{align*}
and we denote $\omega=\omega_idx^i$, $\omega^\sharp=g^{jk}\omega_k\partial_j$.
Clearly musical isomorphism can be extended to arbitrary tensors.

\subsection{Contration}

We discuss contraction of two indices of a tensor in this subsection.

First, let us check this naive example.
Let $V$ be an $n$-dimensional Euclidean space with flat metric (i.e.\ with metric $\delta_{ij}$), $S:V\to V$ be a (symmetric) linear transformation, $L$ be its associated bilinear function.
Let $S$ and $L$ has matrices
\[\begin{bmatrix}
    a^1_1 & a^1_2 & \cdots & a^1_n \\
    a^2_1 & a^2_2 & \cdots & a^2_n \\
    \vdots & \vdots & \ddots & \vdots \\
    a^n_1 & a^n_2 & \cdots & a^n_n
\end{bmatrix}\quad\text{and}\quad\begin{bmatrix}
    a_{11} & a_{12} & \cdots & a_{1n} \\
    a_{21} & a_{22} & \cdots & a_{2n} \\
    \vdots & \vdots & \ddots & \vdots \\
    a_{n1} & a_{n2} & \cdots & a_{nn}
\end{bmatrix},\]
the matrices of $S$ and $L$ are related by the musical isomorphism of Euclidean metric $\delta_{ij}$, since we know $a^i_j=a_{ij}$.
Clearly we want their trace or contraction to be the same.
To define the contraction of $S$ is relatively easy: $S$ has expression
\[S=a^i_jv^j\otimes v_i^*,\]
where we take $\{v^i\},\{v_i^*\}$ to be a basis and whose dual basis of $V$ respectively.
Plug $v^j$ into $v_i^*$ and take summation, we obtain
\[\tr_{1,2}S=\sum_{i=1}^na^i_i.\]
Since we want
\[\tr_{1,2}L=\sum_{i=1}^na_{ii},\]
this enlighten us that two tensors should have same contraction modulo a musical isomorphism.

Thus we have the following definition.
\begin{defn}
    Let $(M,g)$ be a Riemannian manifold, $(x^1,\cdots,x^n)$ be a local coordinate.
    Let $S=S^i_j\partial_i\otimes dx^j$ be an $(1,1)$-tensor, then the \emph{contraction} of indices $1,2$ is defined to be
    \[\tr_{1,2}S=S^i_i.\]
    Let $L=L_{ij}dx^i\otimes dx^j$ be an $(0,2)$-tensor, then the contraction of indices $1,2$ is defined to be
    \[\tr_{1,2}L=g^{ij}L_{ij}.\]
    Similarly, we can define the contraction of tensors of type $(r,s)$.
\end{defn}

\subsection{Norm of a Tensor}

We still look at our very first definition of norm.
We use an exaggerated way to write the norm of a vector field $X$:
\begin{align*}
    |X|^2&=g(X,X)\\
    &=\tr_{1,2}\left(\tr_{1,3}(g\otimes X)\otimes X\right)\\
    &=\tr_{1,2}(X^\flat\otimes X).
\end{align*}
If we think $X:T^*M\to\mathbb{R}$ is a function on $T^*M$, then $X^\flat:TM\to\mathbb{R}$ is its adjoint, which can be denoted by $X^*$.
Thus we reached the definition.

\begin{defn}
    Let $(M,g)$ be a Riemannian manifold, and $T$ be an $(r,s)$-tensor on $M$.
    Let $T^*$ be the $(s,r)$-tensor related to $T$ with musical isomorphism, we permute $T^*$ with covariant part lying before contravariant part.
    Define the \emph{norm} of $T$ by
    \[|T|^2=\tr_{1,r+s+1}\tr_{2,r+s+2}\cdots\tr_{r+s,2r+2s}(T^*\otimes T).\]
\end{defn}

\begin{eg}
    We will use the norm of a Hessian, that is, a $(0,2)$-tensor.
    Let $L=L_{ij}dx^i\otimes dx^j$ be such a tensor.
    Then we have
    \[L^*=g^{ik}g^{jl}L_{kl}\partial_i\otimes\partial_j,\]
    and
    \[L^*\otimes L=g^{im}g^{jn}L_{mn}L_{kl}\partial_i\otimes\partial_j\otimes dx^k\otimes dx^l.\]
    We take contraction and obtain
    \[\tr_{1,3}\tr_{2,4}(L^*\otimes L)=g^{ik}g^{jl}L_{ij}L_{kl}.\]
\end{eg}

\subsection{Covariant Derivative and Covariant Differentiation}

Let $(M,g)$ be a Riemannian manifold and $\nabla$ be its Levi--Civita connection.
Recall that given a vector field $Y$ and a tangent vector $X_p\in T_pM$ (let it be the restriction of $X$ at $p$), the covariant derivative $\nabla_XY(p)$ can be evaluated as follows:
Choose an arbitrary curve $\gamma:I\to M$ with $\gamma(0)=p,\dot\gamma(0)=X_p$, let $P_t$ be the parallel transportation of $\nabla$ along $\gamma$.
Then we have
\[\nabla_XY|_p=\lim_{t\to 0}\frac{1}{t}\left(P^{-1}_t(Y(\gamma(t))-Y(\gamma(0)))\right).\]

Generalize this idea to tensors, just as we did for Lie derivative, we can reach the definition of covariant derivative of tensors.
\begin{defn}
    Let $T$ be an $(r,s)$-tensor, $p\in M$ and $X_p\in T_pM$ (let it be the restriction of $X$ at $p$).
    We define $\nabla_XT(p)$ as follows:
    Choose an arbitrary curve $\gamma:I\to M$ with $\gamma(0)=p,\dot\gamma(0)=X_p$, let $P_t$ be the parallel transportation of $\nabla$ along $\gamma$.
    Define $P^\otimes_t:T^{(r,s)}T_{p}M\to T^{(r,s)}T_{\gamma(t)}M$ by
    \[P^\otimes_t=\underbrace{P_t\otimes\cdots\otimes P_t}_{r}\otimes\underbrace{(P^*_t)^{-1}\otimes\cdots\otimes(P^*_t)^{-1}}_{s},\]
    and then we define
    \[\nabla_XT|_p=\lim_{t\to 0}\frac{1}{t}\left((P^\otimes_t)^{-1}(T(\gamma(t)))-T(\gamma(0))\right).\]
\end{defn}

We also have the notion of covariant differentiation.
\begin{defn}
    Let $T$ be an $(r,s)$-tensor.
    The \emph{covariant differentiation} $\nabla T$ of $T$ is an $(r,s+1)$-tensor such that
    \[\nabla T(\cdots,X)=\nabla_XT(\cdots).\]
\end{defn}

\begin{eg}
    The metric compatibility of Levi--Civita connection is equivalent to $\nabla g=0$.
    It's hard to verify this property by now, but we will soon figure out how to compute covariant derivative.
\end{eg}

Two properties are essential to compute the covariant derivative.
We write a lemma for this.

\begin{lem}
    \begin{enumerate}[\rm(1)]
        \item Covariant derivative satisfies the \emph{Leibniz law}, that is, for tensors $S,T$, we have
        \[\nabla_X(S\otimes T)=S\otimes(\nabla_XT)+(\nabla_XS)\otimes T.\]
        \item Covariant derivative commutes with contraction, that is, for tensor $T$, we have
        \[\nabla_X(\tr_{i,j}T)=\tr_{i,j}\nabla_XT.\]
    \end{enumerate}
\end{lem}
\begin{proof}
    For simplicity, we prove for tensor $X\otimes\omega$, a $(1,1)$-tensor.
    General case is similar.
    Choose a curve $\gamma:I\to M$, $\dot\gamma(0)=v$, and a parallel basis $\{e_i(t)\}$ along $\gamma$.
    Let $\{\alpha^i(t)\}$ be the dual basis with respective to $\{e^i(t)\}$, and let
    \begin{align*}
        X(\gamma(t))&=X^i(t)e_i(t),\\
        \omega(\gamma(t))&=\omega_i(t)\alpha^i(t).
    \end{align*}
    Thus we have
    \begin{align*}
        \nabla_v(X\otimes\omega)&=\left.\frac{d}{dt}\right|_{t=0}(X^i(t)\omega_j(t))e_i(0)\otimes\alpha^j(0)\\
        &=\left(\dot{X}^i(0)\omega_j(0)+X^i(0)\dot\omega_j(0)\right)e_i(0)\otimes\alpha^j(0)\\
        &=(\nabla_vX)\otimes\omega+X\otimes(\nabla_v\omega).
    \end{align*}
    Moreover, we have
    \begin{align*}
        \nabla_v(\tr_{1,2}X\otimes\omega)&=\left.\frac{d}{dt}\right|_{t=0}(X^i(t)\omega_i(t))\\
        &=\dot{X}^i(0)\omega_i(0)+X^i(0)\dot\omega_i(0)\\
        &=\tr_{1,2}(\nabla_vX)\otimes\omega+\tr_{1,2}X\otimes(\nabla_v\omega)\\
        &=\tr_{1,2}(\nabla_v(X\otimes\omega)).
    \end{align*}
    Thus we proved the lemma.
\end{proof}

\begin{eg}
    In this example, we illustrate how to calculate the covariant derivative of a covariant tensor.
    Let $T$ be a $(0,s)$-tensor, we want to know what is
    \[(\nabla_XT)(X_1,\cdots,X_s).\]
    For simplicity we let $s=2$, there is no difference for general case.
    As we did before, we write $T(X_1,X_2)$ into a form of contraction, and then use the commutativity of contraction and covariant derivative.
    It writes
    \begin{align*}
        XT(X_1,X_2)=&\nabla_X(T(X_1,X_2))\\
        =&\nabla_X(\tr_{1,3}\tr_{2,4}T\otimes X_1\otimes X_2)\\
        =&\tr_{1,3}\tr_{2,4}\nabla_X(T\otimes X_1\otimes X_2)\\
        =&\tr_{1,3}\tr_{2,4}(T\otimes(\nabla_XX_1)\otimes X_2+T\otimes X_1\otimes(\nabla_XX_2))\\
        &+\tr_{1,3}\tr_{2,4}((\nabla_XT)\otimes X_1\otimes X_2)\\
        =&(\nabla_XT)(X_1,X_2)+T(\nabla_XX_1,X_2)+T(X_1,\nabla_XX_2),
    \end{align*}
    thus we have
    \begin{equation}
        (\nabla_XT)(X_1,X_2)=XT(X_1,X_2)-T(\nabla_XX_1,X_2)-T(X_1,\nabla_XX_2).\label{eq:covdiff eq1}
    \end{equation}
    In particular, if we take $T=g$, then the equation~\eqref{eq:covdiff eq1}~is nothing but the metric compatibility of Levi--Civita connection.
\end{eg}

\begin{eg}
    Sometimes calculating covariant derivative in a local chart is useful.
    Let $\nabla dx^i=\omega_{jk}dx^j\otimes d^k$, then we have
    \begin{align*}
        \omega_{jk}&=(\nabla dx^i)(\partial_j,\partial_k)\\
        &=(\nabla_kdx^i)(\partial_j)\\
        &=\partial_k(dx^i(\partial_j))-dx^i(\nabla_k\partial_j)\\
        &=-\Gamma_{kj}^i.
    \end{align*}
    Thus we have $\nabla dx^i=-\Gamma_{kj}^idx^j\otimes dx^k$.
\end{eg}

\subsection{Curvature Endomorphism}

We next consider second covariant differentiation.
We first introduce a symbol.

\begin{defn}
    Let $T$ be a tensor, $X,Y$ be vector fields, we use $\nabla_{X,Y}^2T$ to denote the tensor
    \[\nabla^2_{X,Y}(\cdots):=\nabla(\nabla T)(\cdots,Y,X).\]
\end{defn}

We have an explicit formula for second covariant differentiation.

\begin{lem}\label{lem:second covdiff}
    We have
    \[\nabla_{X,Y}^2T=\nabla_X\nabla_YT-\nabla_{\nabla_XY}T.\]
\end{lem}

We leave this lemma as an exercise.
(Just remember how do you calculate Hessian in Riemannian geometry class.)

\begin{defn}
    The \emph{curvature endomorphism} is given by
    \begin{align*}
        R(X,Y):\Gamma\left(T^{(r,s)}TM\right)&\to\Gamma\left(T^{(r,s)}TM\right)\\
        T&\mapsto\nabla^2_{Y,X}T-\nabla^2_{X,Y}T.
    \end{align*}
\end{defn}

\begin{prop}[Ricci identity]
    We have
    \[R(X,Y)T=-\nabla_X\nabla_YT+\nabla_Y\nabla_XT+\nabla_{[X,Y]}T.\]
\end{prop}

\begin{rem}
    There are many ways to interpret curvature.
    One way (maybe the most common way) to understand curvature is that curvature measures the deviation of map $X\mapsto\nabla_X$ from being a Lie algebra homomorphism, as the Ricci identity indicates.
    Another way is that curvature measures the deviation of second covariant derivative from being commutative.
    However, this viewpoint cannot explain the existence of $\nabla_{[X,Y]}$ term.
    We choose here the viewpoint of curvature measures the deviation of second covariant differentiation being commutative.

    Last but not least, beware of our sign convention.
\end{rem}
% !TeX root = main.tex

\centerline{\sectionfont\the\sectcount.\ Comparison Inequalities}
\medskip
\noindent{\bf\the\sectcount.\the\subsectcount.\ Bochner's Formula.}
In this section, we first introduce a useful tool, namely Bochner's formula.
\medskip
\proclaim Theorem~\propnumber~{\rm(Bochner's formula)}.
Let $(M,g)$ be a Riemannian manifold, $f$ be a smooth function on $M$, then we have
$${1\over2}\Delta|\nabla f|^2=|\Hess f|^2+\langle\nabla\Delta f,\nabla f\rangle+\Ric(\nabla f,\nabla f).$$\par
\noindent{\it Proof.}\/ Since covariant differentiation is clearly commutative with musical isomorphism, we may use musical isomorphism to obtain Bochner's formula for $1$-form:
$${1\over2}\Delta|df|^2=|\nabla df|^2+\langle\nabla f,\nabla\Delta f\rangle+\Ric(\nabla f,\nabla f).$$
Slightly change Ricci identity we obtain
$$\nabla^2_{X,Y}df(Z)-\nabla^2_{Y,X}df(Z)=df(R(X,Y)Z).\eqdef{eq:Bochner 1}$$
Notice that
$$\nabla^2_{X,Y}df(Z)=\nabla_X(\Hess{f(Z,Y)})=\nabla_X(\Hess{f(Y,Z)})=\nabla^2_{X,Z}df(Y),$$
then by contracting $2,3$ indices, we obtain
$$\tr_{2,3}\nabla^2df(Y,\cdot,\cdot)-\nabla_Y\Delta f=\Ric(Y,\Delta f).\eqdef{eq:Bochner 2}$$
Clearly $\nabla_Y\Delta f=\langle Y,\nabla\Delta f\rangle$.
We evaluate the first term.
Consider $\nabla^2_{X,Y}|df|^2$, we have
$$\nabla^2_{X,Y}|df|^2=2\langle\nabla_Xdf,\nabla_Ydf\rangle+2\langle\nabla_X\nabla_Ydf,df\rangle-2\langle\nabla_{\nabla_XY}df,df\rangle.$$
Contracting $X,Y$, we obtain
$${1\over2}\Delta|df|^2=|\nabla df|^2+\tr_{2,3}\nabla^2df(\nabla f,\cdot,\cdot).$$
Take $Y=\nabla f$ in~\eqref{eq:Bochner 2}, we obtain equation~\eqref{eq:Bochner 1}.
\advance\propcount by 1
\qed
\noindent{\it Remark~\propnumber.}\/ For further computation, we notice that $|\Hess f|^2$ can usually be computed by the sum of squares of eigenvalues of $\Hess f$.
\advance\propcount by 1
\advance\subsectcount by 1
\bigskip
\noindent{\bf\the\sectcount.\the\subsectcount.\ Some Computations.}
We are concerned about volume form of geodesic balls and mean curvature of geodesic spheres, they are connected by the Hessian of distance functions.
For this, we do some calculations.
\medskip
\proclaim Lemma~\propnumber. Let $(M,g)$ be a Riemannian manifold, $r$ be a distance function with respective to a point $p$.
Then within cut locus of $p$, $\Hess{r}=\II_{\partial B_p(\rho)}$, with the normal vector field to be $\nabla r$.\par
\noindent{\it Proof.}\/ First we notice that by Gauss lemma, under geodesic polar coordinate, the metric $g$ has form
$$g=dr^2+g_{ij}d\theta^i\otimes d\theta^j,$$
hence $\nabla r$ is indeed a normal vector field.
Let $X,Y\in T_q\partial B_p(\rho)$, we have
$$\eqalign{
	\Hess{r}(X,Y)&=XYr-(\nabla_XY)r\cr
	&=X\langle Y,\nabla r\rangle-\langle\nabla_XY,\nabla r\rangle\cr
	&=X\langle Y,\nabla\rangle-X\langle Y,\nabla r\rangle+\langle Y,\nabla_X\nabla r\rangle\cr
	&=\langle S(X),Y\rangle,
}$$
where $S$ is the shape operator.
Hence we have $\Hess{r}=\II$.
\advance\propcount by 1
\qed
\noindent{\it Remark~\propnumber.}\/ In contrast to some sign convention, we simply ask second fundamental form and shape operator are associated bilinear form and linear transformation, we don't ask they differ a minus sign.
\advance\propcount by 1
\medskip
By taking contraction, we obtain the following
\medskip
\definexref{cor:mean curv laplacian}{\propnumber}{cor}
\proclaim Corollary~\propnumber. We have $\Delta r=m$, where $m$ is the mean curvature of $\partial B_p(\rho)$.\par
\advance\propcount by 1
Let the volume form of $B_p(\rho)$ be
$$d\vol{}={\cal A}(r,\theta)dr\wedge d\theta^1\wedge\cdots\wedge d\theta^{n-1}.\eqdef{eq:volume form}$$
Then we have
\medskip
\definexref{lem:mean curv}{\propnumber}{lem}
\proclaim Lemma~\propnumber. Let ${\cal A}'$ be the derivative of $\cal A$ with respective to $r$, then
$${{\cal A}'(r,\theta)\over{\cal A}(r,\theta)}=m(r).$$\par
\noindent{\it Proof.}\/ Let's just calculate.
We have
$$\eqalign{
	{{\cal A}'(r,\theta)\over{\cal A}(r,\theta)}&={d\over dr}\log{\cal A}(r,\theta)\cr
	&={d\over dr}\log\sqrt{\det(g_{ij})}\cr
	&={1\over2}{1\over\det(g_{ij})}\det(g_{ij})g^{ij}\partial_r\langle\partial_i,\partial_j\rangle\cr
	&=g^{ij}\langle\nabla_r\partial_i,\partial_j\rangle\cr
	&=g^{ij}\langle\nabla_i\nabla r,\partial_j\rangle\cr
	&=m(r).}$$
\advance\propcount by 1
\advance\subsectcount by 1
\qed
\noindent{\bf\the\sectcount.\the\subsectcount.\ Comparison Inequalities.}
In this subsection we introduce comparison inequalities.
The first one is the mean curvature comparison.
We denote $B^H(\rho)$ the geodesic ball of radius $\rho$ in the space form $S^2(H)$ of constant curvature $H$.
\medskip
\definexref{thm:mean curv}{\propnumber}{thm}
\proclaim Theorem~\propnumber~{\rm(Mean curvature comparison)}.
Let $(M,g)$ be a Riemannian manifold with $\Ric\geq(n-1)H$, then along any minimal geodesic segement from $p$, we have
$$m(\rho)\leq m_H(\rho),$$
where $m_H(\rho)$ denotes the mean curvature of $\partial B^H(\rho)$.\par
\noindent{\it Proof.}\/ Plug $f=r$ into Bochner's formula, notice that $|\nabla r|=1$, we obtain
$$0=|\Hess r|^2+\langle\nabla r,\nabla m\rangle+\Ric(\nabla r,\nabla r).$$
By Cauchy--Schwarz inequality, we have
$$|\Hess r|^2=|\II{}|^2\geq{m^2\over n-1}.$$
Moreover, we have
$$\Ric(\nabla r,\nabla r)\geq(n-1)H.$$
Thus we obtain
$$m'=\langle\nabla r,\nabla m\rangle\leq-{m^2\over n-1}-(n-1)H.$$
For space form, the equality holds, that is
$$m'_H=-{m_H^2\over n-1}-(n-1)H.$$
Since $\lim_{r\to 0}(m-m_H)=0$, by standard Riccati equation comparison, we then obtain the result.
\advance\propcount by 1
\qed
The second theorem is auxiliary, but we still call it a theorem.
\medskip
\definexref{thm:Ricci comparison}{\propnumber}{thm}
\proclaim Theorem~\propnumber~{\rm(Ricci comparison)}.
Let $(M,g)$ be a Riemannian manifold with $\Ric\geq(n-1)H$.
Let ${\cal A}(r,\theta)$ be as in~\eqref{eq:volume form}, ${\cal A}_H(r,\theta)$ be similar for space form $S^2(H)$.
Then along any minimal geodesic segement from $p$,
$${\cal A}(r,\theta)\over{\cal A}_H(r,\theta)\eqdef{eq:Ricci comparison}$$
is nonincreasing with respective to $r$.\par
\noindent{\it Proof.}\/ Notice that the logarithm derivative for~\eqref{eq:Ricci comparison}~is
$$\eqalign{
	{d\over dr}\log\left({{\cal A}(r,\theta)\over{\cal A}_H(r,\theta)}\right)&={{\cal A}'(r,\theta)\over{\cal A}(r,\theta)}-{{\cal A}'_H(r,\theta)\over {\cal A}_H(r,\theta)}\cr
	&=m(r)-m_H(r)\cr
	&\leq 0,
}$$
here we used~\ref{lem:mean curv}~and mean curvature comparison inequality.
\advance\propcount by 1
\qed
Our last theorem for this section is Bishop--Gromov volume comparison inequality.
This is the core of our seminar.
\proclaim Theorem~\propnumber~{\rm(Volume comparison)}.
Let $(M,g)$ be a Riemannian manifold with $\Ric\geq(n-1)H$, then
$$\vol(B_p(\rho))\over\vol(B^H(\rho))$$
is nonincreasing with respective to $\rho$.\par
\noindent{\it Proof.}\/ We have
$$\eqalign{
	\vol(B_p(\rho))&=\int_{S^{n-1}}\int_0^\rho{\cal A}(r,\theta)dr\wedge d\theta^1\wedge\cdots\wedge d\theta^{n-1},\cr
	\vol(B^H(\rho))&=\int_{S^{n-1}}\int_0^\rho{\cal A}_H(r,\theta)dr\wedge d\theta^1\wedge\cdots\wedge d\theta^{n-1}.
}$$
Take derivative with respective to $\rho$, we have
$$\eqalign{
	{d\over d\rho}\left({\vol(B_p(\rho))\over\vol(B^H(\rho))}\right)&={\left(\int_{S^{n-1}}{\cal A}(\rho,\theta)d\vol_{S^{n-1}}\right)\left(\int_{S^{n-1}}\int_0^\rho{\cal A}_H(r,\theta)dr\wedge d\vol_{S^{n-1}}\right)\over(\vol(B^H(\rho))^2}\cr
	&\quad-{\left(\int_{S^{n-1}}{\cal A}_H(\rho,\theta)d\vol_{S^{n-1}}\right)\left(\int_{S^{n-1}}\int_0^\rho{\cal A}(r,\theta)dr\wedge d\vol_{S^{n-1}}\right)\over(\vol(B^H(\rho)))^2}\cr
	&=(\vol(B^H(\rho))^{-2}\int_0^\rho\biggl(\left(\int_{S^{n-1}}{\cal A}(\rho,\theta)d\vol_{S^{n-1}}\right)\left(\int_{S^{n-1}}{\cal A}_H(r,\theta)d\vol_{S^{n-1}}\right)\cr
	&\quad-\left(\int_{S^{n-1}}{\cal A}_H(\rho,\theta)d\vol_{S^{n-1}}\right)\left(\int_{S^{n-1}}{\cal A}(r,\theta)d\vol_{S^{n-1}}\right)\biggr)dr.
}$$
Therefore, to check
$$\rho\mapsto{\vol(B_p(\rho))\over\vol(B^H(\rho))}$$
is nonincreasing, it's suffice to check
$$r\mapsto{\int_{S^{n-1}}{\cal A}(r,\theta)d\vol_{S^{n-1}}\over\int_{S^{n-1}}{\cal A}_H(r,\theta)d\vol_{S^{n-1}}}$$
is nonincreasing.
But we have
$${\int_{S^{n-1}}{\cal A}(r,\theta)d\vol_{S^{n-1}}\over\int_{S^{n-1}}{\cal A}_H(r,\theta)d\vol_{S^{n-1}}}={1\over\omega_{n-1}}\int_{S^{n-1}}{{\cal A}(r,\theta)\over{\cal A}_H(r,\theta)}d\vol_{S^{n-1}},$$
which is nonincreasing by~\ref{thm:Ricci comparison}.
\advance\propcount by 1
\advance\subsectcount by 1
\qed
\noindent{\bf\the\sectcount.\the\subsectcount.\ Comparison Inequality in Weak Sense.}
Sometimes we will deal with functions with bad smoothness, so we need comparison theorems in weak sense.
Here we introduce a special case of Laplacian comparison inequality in distribution sense.\par
First, notice by~\ref{cor:mean curv laplacian}, we have
\medskip
\proclaim Proposition~\propnumber.
Under same assumption of~\ref{thm:mean curv}, mean curvature comparison inequality is equivalent to the following\/ {\rm Laplacian comparison inequality}
$$\Delta r\leq\Delta r_H.$$
\advance\propcount by 1
\par
\noindent{\bf Example~\propnumber.}
In our seminar, we are concerned about $H=0$ case the most.
We haven't computed $\Hess r$ for ${\bb R}^n$ yet.
Let's compute for $p=0$ and $\Hess r(\rho)$.
This is just the second fundamental form of sphere
$$(x^1)^2+(x^2)^2+\cdots+(x^n)^2=\rho^2.$$
Fix the outward normal vectot field $N={1\over\rho}{\bf x}$, then we have
$$\II(X,Y)=\langle\nabla_XN,Y\rangle=\left\langle{1\over\rho}X{\bf x},Y\right\rangle={1\over\rho}\langle X,Y\rangle.$$
Hence we have
$$\Hess r={1\over r}(g-dr\otimes dr).$$
Take contraction we have
$$\Delta r={n-1\over r}.$$
\advance\propcount by 1
\medskip
We avoided using Jacobi field to compute Hessian of distance function.
This method is limited, and is not recommended.
\medskip
\proclaim Theorem~\propnumber~{\rm(Laplacian comparison)}~.
Let $(M,g)$ be a Riemannian manifold with $\Ric\geq 0$.
Then for the distance function to $p$, the inequality
$$\Delta r\leq{n-1\over r}$$
holds in distribution sense, that is, for any $\varphi\in C^\infty_0(M),\ \varphi\geq 0$, we have
$$\int_Mr\Delta\varphi\leq\int_M{n-1\over r}\cdot\varphi.$$
\par
\noindent{\it Proof.}\/ Let $M$ be decomposed into cut locus with respective to $p$ and a star-shaped domain, namely $M=:\Omega\sqcup{\rm Cut}(p)$.
Lipschitz function $r$ is differentiable in $\Omega$, thus within $\Omega$ we have
$$\Delta r\leq{n-1\over r}.$$
Fix $\varphi\in C^\infty_0(M),\ \varphi\geq 0$.
Since $|{\rm Cut}(p)|=0$, we have
$$\int_Mr\Delta\varphi=\int_\Omega r\Delta\varphi.$$
Since $\Omega$ is star-shaped, we can choose an increasing sequence $\Omega_k\subset\Omega$ such that
$$\lim_{k\to\infty}\Omega_k=\Omega,$$
and each $\Omega_k$ is obtained by shrinking $\Omega$ along $r$'s direction.
Since Stokes' formula is valid for Lipschitz functions, and $\varphi$ has compact support, we have
$$\eqalign{
    \int_\Omega r\Delta\varphi&=-\int_\Omega\langle\nabla\varphi,\nabla r\rangle\cr
    &=-\lim_{k\to\infty}\int_{\Omega_k}\langle\nabla\varphi,\nabla r\rangle.
}$$
Last equality holds since $|\nabla r|=1$ within $\Omega$ and $\nabla\varphi$ is bounded, then apply Lebesgue's dominated convergence theorem.
By Green's formula, we have
$$-\int_{\Omega_k}\langle\nabla r,\nabla\varphi\rangle=\nabla_{\Omega_k}\Delta r\cdot\varphi-\int_{\partial\Omega_k}\varphi\cdot{\partial r\over\partial\nu},$$
where $\nu$ is the unit outer normal vector field.
Since $\Omega_k$ is obtained by shrinking $\Omega$ along $r$'s direction and $\varphi\geq 0$, we have
$$\int_{\partial\Omega_k}\varphi\cdot{\partial r\over\partial\nu}\geq 0.$$
Thus we have
$$\eqalign{
    -\int_{\Omega_k}\langle\nabla r,\nabla\varphi\rangle&\leq\int_{\Omega_k}\Delta r\cdot\varphi\cr
    &\leq\int_{\Omega_k}{n-1\over r}\cdot\varphi.
}$$
Finally we have
$$\eqalign{
    \int_Mr\Delta\varphi&\leq\lim_{k\to\infty}\int_{\Omega_k}{n-1\over r}\cdot\varphi\cr
    &=\int_\Omega{n-1\over r}\cdot\varphi\cr
    &=\int_M{n-1\over r}\cdot\varphi,
}$$
we used Lebesgue's dominated convergence theorem and $|{\rm Cut}(p)|=0$ again here.
\advance\propcount by 1
\qed
\subsectcount=1
\propcount=1
\eqnumber=0
\advance\sectcount by 1
\bigskip


\end{document}
\documentclass{amsart}

\usepackage{amsmath}
\usepackage{amssymb}
\usepackage{amsthm}
\usepackage[shortlabels]{enumitem}
\usepackage[colorlinks]{hyperref}

\theoremstyle{definition}
\newtheorem{defn}{Definition}[section]
\newtheorem{eg}[defn]{Example}
\theoremstyle{plain}
\newtheorem{thm}[defn]{Theorem}
\newtheorem{prop}[defn]{Proposition}
\newtheorem{lem}[defn]{Lemma}
\newtheorem{cor}[defn]{Corollary}
\theoremstyle{remark}
\newtheorem{rem}[defn]{Remark}

\numberwithin{equation}{section}

\newcommand{\R}[4]{R_{#1}{}_{#2}{}_{#3}{}^{#4}}
\DeclareMathOperator{\vol}{vol}
\DeclareMathOperator{\Sect}{Sect}
\DeclareMathOperator{\Ric}{Ric}
\DeclareMathOperator{\Scal}{Scal}
\DeclareMathOperator{\tr}{tr}
\DeclareMathOperator{\Hess}{Hess}
\DeclareMathOperator{\II}{II}
\DeclareMathOperator{\Cut}{Cut}
\DeclareMathOperator{\sn}{sn}

\title[Ricci Rigidity Notes]{Notes on Rigidity Results for Manifolds with Lower Ricci Curvature Bound}
\author{Mengchen Zeng}
\date{\today}
\address{School of Mathematical Sciences, Beijing Normal University, Beijing, China, P.\ R., 100875.}

\begin{document}

\maketitle

In this seminar, all Riemannian manifolds are assumed to be complete.
The manifold $M$ is understood to be $n$-dimensional if not specially mentioned.

\section{Tensors}

\subsection{Notation}

We first fix the notation of tensors.
Let $(M,g)$ be a Riemannian manifold.
A \emph{tensor} $T$ of type $(r,s)$ is a smooth section of vector bundle
\[T^{(r,s)}(TM)=\left(\bigotimes_{i=1}^rTM\right)\otimes\left(\bigotimes_{j=1}^sT^*M\right).\]
On each fiber, a tensor can be regarded as a multilinear map
\[T|_{T_pM}:\underbrace{T^*_pM\times\cdots\times T^*_pM}_{r}\times\underbrace{T_pM\times\cdots\times T_pM}_{s}\to\mathbb{R},\]
so we can talk about symmetric and positive-definite tensor (for $(0,2)$-tensor), alternating tensor (for $(0,n)$-tensor) well.
Tensors of type $(r,0)$ are called contravariant tensors, and of type $(0,s)$ are called covariant tensors.

Let $(x^1,\cdots,x^n)$ be a local coordinate.
We will adopt the notation $\{\partial_1,\cdots,\partial_n\}$ for a local frame for $TM$, and $\{dx^1,\cdots,dx^n\}$ for a local frame for $T^*M$.
We will adopt the \textsc{Einstein} summation convention, so the local expression for an $(r,s)$-tensor $T$ is
\[T=T^{i_1\cdots i_r}_{j_1\cdots j_s}\partial_{i_1}\otimes\cdots\otimes\partial_{i_r}\otimes dx^{j_1}\otimes\cdots\otimes dx^{j_s}.\]

The tangent space $TM$ and cotangent space $T^*M$ are canonically isomorphic via \emph{musical isomorphism}.
Under a local coordinate $(x^1,\cdots,x^n)$, let $g=g_{ij}$.
Then we have the musical isomorphism (``lowering index'')
\begin{align*}
    \flat:TM&\to T^*M\\
    X^i\partial_i&\mapsto g_{jk}X^kdx^j,
\end{align*}
and we denote $X=X^i\partial_i$, $X^\flat=g_{jk}X^kdx^j$.
The inverse is given by (``raising index'')
\begin{align*}
    \sharp:T^*M&\to TM\\
    \omega_idx^i&\mapsto g^{jk}\omega_k\partial_j,
\end{align*}
and we denote $\omega=\omega_idx^i$, $\omega^\sharp=g^{jk}\omega_k\partial_j$.
Clearly musical isomorphism can be extended to arbitrary tensors.

\subsection{Contration}

We discuss contraction of two indices of a tensor in this subsection.

First, let us check this naive example.
Let $V$ be an $n$-dimensional Euclidean space with flat metric (i.e.\ with metric $\delta_{ij}$), $S:V\to V$ be a (symmetric) linear transformation, $L$ be its associated bilinear function.
Let $S$ and $L$ has matrices
\[\begin{bmatrix}
    a^1_1 & a^1_2 & \cdots & a^1_n \\
    a^2_1 & a^2_2 & \cdots & a^2_n \\
    \vdots & \vdots & \ddots & \vdots \\
    a^n_1 & a^n_2 & \cdots & a^n_n
\end{bmatrix}\quad\text{and}\quad\begin{bmatrix}
    a_{11} & a_{12} & \cdots & a_{1n} \\
    a_{21} & a_{22} & \cdots & a_{2n} \\
    \vdots & \vdots & \ddots & \vdots \\
    a_{n1} & a_{n2} & \cdots & a_{nn}
\end{bmatrix},\]
the matrices of $S$ and $L$ are related by the musical isomorphism of Euclidean metric $\delta_{ij}$, since we know $a^i_j=a_{ij}$.
Clearly we want their trace or contraction to be the same.
To define the contraction of $S$ is relatively easy: $S$ has expression
\[S=a^i_jv^j\otimes v_i^*,\]
where we take $\{v^i\},\{v_i^*\}$ to be a basis and whose dual basis of $V$ respectively.
Plug $v^j$ into $v_i^*$ and take summation, we obtain
\[\tr_{1,2}S=\sum_{i=1}^na^i_i.\]
Since we want
\[\tr_{1,2}L=\sum_{i=1}^na_{ii},\]
this enlighten us that two tensors should have same contraction modulo a musical isomorphism.

Thus we have the following definition.
\begin{defn}
    Let $(M,g)$ be a Riemannian manifold, $(x^1,\cdots,x^n)$ be a local coordinate.
    Let $S=S^i_j\partial_i\otimes dx^j$ be an $(1,1)$-tensor, then the \emph{contraction} of indices $1,2$ is defined to be
    \[\tr_{1,2}S=S^i_i.\]
    Let $L=L_{ij}dx^i\otimes dx^j$ be an $(0,2)$-tensor, then the contraction of indices $1,2$ is defined to be
    \[\tr_{1,2}L=g^{ij}L_{ij}.\]
    Similarly, we can define the contraction of tensors of type $(r,s)$.
\end{defn}

\subsection{Norm of a Tensor}

We still look at our very first definition of norm.
We use an exaggerated way to write the norm of a vector field $X$:
\begin{align*}
    |X|^2&=g(X,X)\\
    &=\tr_{1,2}\left(\tr_{1,3}(g\otimes X)\otimes X\right)\\
    &=\tr_{1,2}(X^\flat\otimes X).
\end{align*}
If we think $X:T^*M\to\mathbb{R}$ is a function on $T^*M$, then $X^\flat:TM\to\mathbb{R}$ is its adjoint, which can be denoted by $X^*$.
Thus we reached the definition.

\begin{defn}
    Let $(M,g)$ be a Riemannian manifold, and $T$ be an $(r,s)$-tensor on $M$.
    Let $T^*$ be the $(s,r)$-tensor related to $T$ with musical isomorphism, we permute $T^*$ with covariant part lying before contravariant part.
    Define the \emph{norm} of $T$ by
    \[|T|^2=\tr_{1,r+s+1}\tr_{2,r+s+2}\cdots\tr_{r+s,2r+2s}(T^*\otimes T).\]
\end{defn}

\begin{eg}
    We will use the norm of a Hessian, that is, a $(0,2)$-tensor.
    Let $L=L_{ij}dx^i\otimes dx^j$ be such a tensor.
    Then we have
    \[L^*=g^{ik}g^{jl}L_{kl}\partial_i\otimes\partial_j,\]
    and
    \[L^*\otimes L=g^{im}g^{jn}L_{mn}L_{kl}\partial_i\otimes\partial_j\otimes dx^k\otimes dx^l.\]
    We take contraction and obtain
    \[\tr_{1,3}\tr_{2,4}(L^*\otimes L)=g^{ik}g^{jl}L_{ij}L_{kl}.\]
\end{eg}

\subsection{Covariant Derivative and Covariant Differentiation}

Let $(M,g)$ be a Riemannian manifold and $\nabla$ be its Levi--Civita connection.
Recall that given a vector field $Y$ and a tangent vector $X_p\in T_pM$ (let it be the restriction of $X$ at $p$), the covariant derivative $\nabla_XY(p)$ can be evaluated as follows:
Choose an arbitrary curve $\gamma:I\to M$ with $\gamma(0)=p,\dot\gamma(0)=X_p$, let $P_t$ be the parallel transportation of $\nabla$ along $\gamma$.
Then we have
\[\nabla_XY|_p=\lim_{t\to 0}\frac{1}{t}\left(P^{-1}_t(Y(\gamma(t))-Y(\gamma(0)))\right).\]

Generalize this idea to tensors, just as we did for Lie derivative, we can reach the definition of covariant derivative of tensors.
\begin{defn}
    Let $T$ be an $(r,s)$-tensor, $p\in M$ and $X_p\in T_pM$ (let it be the restriction of $X$ at $p$).
    We define $\nabla_XT(p)$ as follows:
    Choose an arbitrary curve $\gamma:I\to M$ with $\gamma(0)=p,\dot\gamma(0)=X_p$, let $P_t$ be the parallel transportation of $\nabla$ along $\gamma$.
    Define $P^\otimes_t:T^{(r,s)}T_{p}M\to T^{(r,s)}T_{\gamma(t)}M$ by
    \[P^\otimes_t=\underbrace{P_t\otimes\cdots\otimes P_t}_{r}\otimes\underbrace{(P^*_t)^{-1}\otimes\cdots\otimes(P^*_t)^{-1}}_{s},\]
    and then we define
    \[\nabla_XT|_p=\lim_{t\to 0}\frac{1}{t}\left((P^\otimes_t)^{-1}(T(\gamma(t)))-T(\gamma(0))\right).\]
\end{defn}

We also have the notion of covariant differentiation.
\begin{defn}
    Let $T$ be an $(r,s)$-tensor.
    The \emph{covariant differentiation} $\nabla T$ of $T$ is an $(r,s+1)$-tensor such that
    \[\nabla T(\cdots,X)=\nabla_XT(\cdots).\]
\end{defn}

\begin{eg}
    The metric compatibility of Levi--Civita connection is equivalent to $\nabla g=0$.
    It's hard to verify this property by now, but we will soon figure out how to compute covariant derivative.
\end{eg}

Two properties are essential to compute the covariant derivative.
We write a lemma for this.

\begin{lem}
    \begin{enumerate}[\rm(1)]
        \item Covariant derivative satisfies the \emph{Leibniz law}, that is, for tensors $S,T$, we have
        \[\nabla_X(S\otimes T)=S\otimes(\nabla_XT)+(\nabla_XS)\otimes T.\]
        \item Covariant derivative commutes with contraction, that is, for tensor $T$, we have
        \[\nabla_X(\tr_{i,j}T)=\tr_{i,j}\nabla_XT.\]
    \end{enumerate}
\end{lem}
\begin{proof}
    For simplicity, we prove for tensor $X\otimes\omega$, an $(1,1)$-tensor.
    General case is similar.
    Choose a curve $\gamma:I\to M$, $\dot\gamma(0)=v$, and a parallel basis $\{e_i(t)\}$ along $\gamma$.
    Let $\{\alpha^i(t)\}$ be the dual basis with respective to $\{e^i(t)\}$, and let
    \begin{align*}
        X(\gamma(t))&=X^i(t)e_i(t),\\
        \omega(\gamma(t))&=\omega_i(t)\alpha^i(t).
    \end{align*}
    Thus we have
    \begin{align*}
        \nabla_v(X\otimes\omega)&=\left.\frac{d}{dt}\right|_{t=0}(X^i(t)\omega_j(t))e_i(0)\otimes\alpha^j(0)\\
        &=\left(\dot{X}^i(0)\omega_j(0)+X^i(0)\dot\omega_j(0)\right)e_i(0)\otimes\alpha^j(0)\\
        &=(\nabla_vX)\otimes\omega+X\otimes(\nabla_v\omega).
    \end{align*}
    Moreover, we have
    \begin{align*}
        \nabla_v(\tr_{1,2}X\otimes\omega)&=\left.\frac{d}{dt}\right|_{t=0}(X^i(t)\omega_i(t))\\
        &=\dot{X}^i(0)\omega_i(0)+X^i(0)\dot\omega_i(0)\\
        &=\tr_{1,2}(\nabla_vX)\otimes\omega+\tr_{1,2}X\otimes(\nabla_v\omega)\\
        &=\tr_{1,2}(\nabla_v(X\otimes\omega)).
    \end{align*}
    Thus we proved the lemma.
\end{proof}

\begin{eg}
    In this example, we illustrate how to calculate the covariant derivative of a covariant tensor.
    Let $T$ be a $(0,s)$-tensor, we want to know what is
    \[(\nabla_XT)(X_1,\cdots,X_s).\]
    For simplicity we let $s=2$, there is no difference for general case.
    As we did before, we write $T(X_1,X_2)$ into a form of contraction, and then use the commutativity of contraction and covariant derivative.
    It writes
    \begin{align*}
        XT(X_1,X_2)=&\nabla_X(T(X_1,X_2))\\
        =&\nabla_X(\tr_{1,3}\tr_{2,4}T\otimes X_1\otimes X_2)\\
        =&\tr_{1,3}\tr_{2,4}\nabla_X(T\otimes X_1\otimes X_2)\\
        =&\tr_{1,3}\tr_{2,4}(T\otimes(\nabla_XX_1)\otimes X_2+T\otimes X_1\otimes(\nabla_XX_2))\\
        &+\tr_{1,3}\tr_{2,4}((\nabla_XT)\otimes X_1\otimes X_1)\\
        =&(\nabla_XT)(X_1,X_2)+T(\nabla_XX_1,X_2)+T(X_1,\nabla_XX_2),
    \end{align*}
    thus we have
    \begin{equation}
        (\nabla_XT)(X_1,X_2)=XT(X_1,X_2)-T(\nabla_XX_1,X_2)-T(X_1,\nabla_XX_2).\label{eq:covdiff eq1}
    \end{equation}
    In particular, if we take $T=g$, then the equation~\eqref{eq:covdiff eq1}~is nothing but the metric compatibility of Levi--Civita connection.
\end{eg}

\begin{eg}
    Sometimes calculating covariant derivative in a local chart is useful.
    Let $\nabla dx^i=\omega_{jk}dx^j\otimes d^k$, then we have
    \begin{align*}
        \omega_{jk}&=(\nabla dx^i)(\partial_j,\partial_k)\\
        &=(\nabla_kdx^i)(\partial_j)\\
        &=\partial_k(dx^i(\partial_j))-dx^i(\nabla_k\partial_j)\\
        &=-\Gamma_{kj}^i.
    \end{align*}
    Thus we have $\nabla dx^i=-\Gamma_{kj}^idx^j\otimes dx^k$.
\end{eg}

\subsection{Curvature Endomorphism}

We next consider second covariant differentiation.
We first introduce a symbol.

\begin{defn}
    Let $T$ be a tensor, $X,Y$ be vector fields, we use $\nabla_{X,Y}^2T$ to denote the tensor
    \[\nabla^2_{X,Y}(\cdots):=\nabla(\nabla T)(\cdots,Y,X).\]
\end{defn}

We have an explicit formula for second covariant differentiation.

\begin{lem}\label{lem:second covdiff}
    We have
    \[\nabla_{X,Y}^2T=\nabla_X\nabla_YT-\nabla_{\nabla_XY}T.\]
\end{lem}

We leave this lemma as an exercise.
(Just remember how do you calculate Hessian in Riemannian geometry class.)

\begin{defn}
    The \emph{curvature endomorphism} is given by
    \begin{align*}
        R(X,Y):\Gamma\left(T^{(r,s)}TM\right)&\to\Gamma\left(T^{(r,s)}TM\right)\\
        T&\mapsto\nabla^2_{Y,X}T-\nabla^2_{X,Y}T.
    \end{align*}
\end{defn}

\begin{prop}[Ricci identity]
    We have
    \[R(X,Y)T=-\nabla_X\nabla_YT+\nabla_Y\nabla_XT+\nabla_{[X,Y]}T.\]
\end{prop}

\begin{rem}
    There are many ways to interpret curvature.
    One way (maybe the most common way) to understand curvature is that curvature measures the deviation of map $X\mapsto\nabla_X$ from being a Lie algebra homomorphism, as the Ricci identity indicates.
    Another way is that curvature measures the deviation of second covariant derivative from being commutative.
    However, this viewpoint cannot explain the existence of $\nabla_{[X,Y]}$ term.
    We choose here the viewpoint of curvature measures the deviation of second covariant differentiation being commutative.

    Last but not least, beware of our sign convention.
\end{rem}
\section{Comparison Inequalities}

\subsection{Bochner's Formula}

In this section, we first introduce a useful tool, namely Bochner's formula.

\begin{thm}[Bochner's formula]
    Let $(M,g)$ be a Riemannian manifold, $f$ be a smooth function on $M$, then we have
    \[\frac{1}{2}\Delta|\nabla f|^2=|\Hess{f}|^2+\langle\nabla\Delta f,\nabla f\rangle+\Ric(\nabla f,\nabla f).\]
\end{thm}
\begin{proof}
    Since covariant differentiation is clearly commutative with musical isomorphism, we may use musical isomorphism to obtain Bochner's formula for $1$-form:
    \begin{equation}
        \frac{1}{2}\Delta|df|^2=|\nabla df|^2+\langle\nabla f,\Delta f\rangle+\Ric(\nabla f,\nabla f).\label{eq:Bochner eq0}
    \end{equation}
    Slightly change Ricci identity we obtain
    \[\nabla^2_{X,Y}df(Z)-\nabla^2_{Y,X}df(Z)=df(R(X,Y)Z).\]
    Notice that
    \[\nabla^2_{X,Y}df(Z)=\nabla_X(\Hess{f}(Z,Y))=\nabla_X(\Hess{f}(Y,Z))=\nabla^2_{X,Z}df(Y),\]
    then by contracting $2,3$ indices, we obtain
    \begin{equation}
        \tr_{2,3}\nabla^2df(Y,\cdot,\cdot)-\nabla_Y\Delta f=\Ric(Y,\nabla f).\label{eq:Bochner eq1}
    \end{equation}
    Clearly $\nabla_Y\Delta f=\langle Y,\nabla\Delta f\rangle$.
    We evaluate the first term.
    Consider $\nabla^2_{X,Y}|df|^2$, we have
    \[\nabla^2_{X,Y}|df|^2=2\langle\nabla_Xdf,\nabla_Ydf\rangle+2\langle\nabla_X\nabla_Ydf,df\rangle-2\langle\nabla_{\nabla_XY}df,df\rangle.\]
    Contracting $X,Y$, we obtain
    \[\frac{1}{2}\Delta|df|^2=|\nabla df|^2+\tr_{2,3}\nabla^2df(\nabla f,\cdot,\cdot)\]
    Take $Y=\nabla f$ in~\eqref{eq:Bochner eq1}, we obtain equation~\eqref{eq:Bochner eq0}.
\end{proof}

\begin{rem}
    For further computation, we notice that $|\Hess{f}|^2$ can usually be computed by the sum of squares of eigenvalues of $\Hess{f}$.
\end{rem}

\subsection{Some Computations}

We are concerned about volume form of geodesic balls and mean curvature of geodesic spheres, they are connected by Hessian of distance functions.
For this, we do some calculations.

\begin{lem}
    Let $(M,g)$ be a Riemannian manifold, $r$ be a distance function with respective to a point $p$.
    Then within cut locus of $p$, $\Hess{r}=\II_{\partial B_p(\rho)}$, with the normal vector field to be $\nabla r$.
\end{lem}
\begin{proof}
    First we notice that by Gauss lemma, under geodesic polar coordinate, the metric $g$ has form
    \[g=dr^2+g_{ij}d\theta^i\otimes d\theta^j,\]
    hence $\nabla r$ is indeed a normal vector field.
    Let $X,Y\in T_{q}\partial B_p(\rho)$, we have
    \begin{align*}
        \Hess{r}(X,Y)&=XYr-\nabla_XYr\\
        &=X\langle Y,\nabla r\rangle-\langle\nabla_XY,\nabla r\rangle\\
        &=X\langle Y,\nabla r\rangle-X\langle Y,\nabla r\rangle+\langle Y,\nabla_X\nabla r\rangle\\
        &=\langle S(X),Y\rangle,
    \end{align*}
    where $S$ is the shape operator.
    Hence we have $\Hess{r}=\II$.
\end{proof}

\begin{rem}
    In contrast to some sign convention, we simply ask second fundamental form and shape operator are associated bilinear form and linear transformation, we don't ask they differ a minus sign.
\end{rem}

By taking contraction, we obtain the following.
\begin{cor}\label{cor:Delta r=m}
    We have $\Delta r=m$, where $m$ is the mean curvature of $\partial B_p(\rho)$.
\end{cor}

Let the volume form of $B_p(\rho)$ be
\begin{equation}
    d\vol=\mathcal{A}(r,\theta)dr\wedge d\theta^1\wedge\cdots\wedge d\theta^{n-1}.\label{eq:volume form}
\end{equation}
Then we have

\begin{lem}\label{lem:mean curv and A}
    Let $\mathcal{A}'$ be the derivative with respective to $r$, then
    \[\frac{\mathcal{A}'(r,\theta)}{\mathcal{A}(r,\theta)}=m(r).\]
\end{lem}
\begin{proof}
    Let's just calculate.
    We have
    \begin{align*}
        \frac{\mathcal{A}'(r,\theta)}{\mathcal{A}(r,\theta)}&=\frac{d}{dr}\log{\mathcal{A}(r,\theta)}\\
        &=\frac{d}{dr}\log{\sqrt{\det(g_{ij})}}\\
        &=\frac{1}{2}\frac{1}{\det(g_{ij})}\det(g_{ij})g^{ij}\partial_r\langle\partial_i,\partial_j\rangle\\
        &=g^{ij}\langle\nabla_r\partial_i,\partial_j\rangle\\
        &=g^{ij}\langle\nabla_i\nabla r,\partial_j\rangle\\
        &=m(r).\qedhere
    \end{align*}
\end{proof}

\subsection{Comparison Inequalities}

In this subsection we introduce comparison inequalities.
The first one is the mean curvature comparison.
We denote $B^H(\rho)$ the geodesic ball of radius $\rho$ in the space form $S^2(H)$ of constant curvature $H$.

\begin{thm}[Mean curvature comparison]\label{thm:mean comparison}
    Let $(M,g)$ be a Riemannian manifold with $\Ric\geq(n-1)H$, then along any minimal geodesic segement from $p$, we have
    \[m(\rho)\leq m_H(\rho),\]
    where $m_H(\rho)$ denotes the mean curvature of $\partial B^H(\rho)$.
\end{thm}
\begin{proof}
    Plug $f=r$ into Bochner's formula, notice that $|\nabla r|=1$, we obtain
    \[0=|\Hess{r}|^2+\langle\nabla r,\nabla m\rangle+\Ric(\nabla r,\nabla r).\]
    By Cauchy--Schwarz inequality, we have
    \[|\Hess{r}|^2=|\II|^2\geq\frac{m^2}{n-1}.\]
    Moreover, we have
    \[\Ric(\nabla r,\nabla r)\geq(n-1)H.\]
    Thus we obtain
    \[m'=\langle\nabla r,\nabla m\rangle\leq-\frac{m^2}{n-1}-(n-1)H.\]
    For space form, the equality holds, that is
    \[m_H'=-\frac{m^2_H}{n-1}-(n-1)H.\]
    Since $\lim_{r\to 0}(m-m_H)=0$, by standard Riccati equation comparison, we then obtain the result.
\end{proof}


The second theorem is auxiliary, but we still call it a theorem.

\begin{thm}[Ricci comparison]\label{thm:Ricci comparison}
    Let $(M,g)$ be a Riemannian manifold with $\Ric\geq(n-1)H$.
    Let $\mathcal{A}(r,\theta)$ be as in~\eqref{eq:volume form}, $\mathcal{A}_H(r,\theta)$ be similar for space form $S^2(H)$.
    Then along any minimal geodesic segement from $p$,
    \begin{equation}
        \frac{\mathcal{A}(r,\theta)}{\mathcal{A}_H(r,\theta)}\label{eq:ricci comparison defn}
    \end{equation}
    is nonincreasing with respective to $r$.
\end{thm}
\begin{proof}
    Notice that the logarithm derivative for~\eqref{eq:ricci comparison defn}~is
    \begin{align*}
        \frac{d}{dr}\log\left(\frac{\mathcal{A}(r,\theta)}{\mathcal{A}_H(r,\theta)}\right)&=\frac{\mathcal{A}'(r,\theta)}{\mathcal{A}(r,\theta)}-\frac{\mathcal{A}'_H(r,\theta)}{\mathcal{A}_H(r,\theta)}\\
        &=m(r)-m_H(r)\\
        &\leq 0,
    \end{align*}
    here we used Lemma~\ref{lem:mean curv and A}~and mean curvature comparison inequality.
\end{proof}

Our last theorem for this section is Bishop--Gromov volume comparison inequality.
This is the core of our seminar.
\begin{thm}[Volume comparison]
    Let $(M,g)$ be a Riemannian manifold with $\Ric\geq(n-1)H$, then
    \[\frac{\vol(B_p(\rho))}{\vol(B^H(\rho))}\]
    is nonincreasing with respective to $\rho$.
\end{thm}
\begin{proof}
    We have
    \begin{align*}
        \vol(B_p(\rho))&=\int_{S^{n-1}}\int_0^\rho\mathcal{A}(r,\theta)dr\wedge d\theta^1\wedge\cdots\wedge d\theta^{n-1}\\
        \vol(B^H(\rho))&=\int_{S^{n-1}}\int_0^\rho\mathcal{A}_H(r,\theta)dr\wedge d\theta^1\wedge\cdots\wedge d\theta^{n-1}.
    \end{align*}
    Take derivative with respective to $\rho$, we have
    \begin{align*}
        &\frac{d}{d\rho}\left(\frac{\vol(B_p(\rho))}{\vol(B^H(\rho))}\right)\\
        =&\frac{\left(\int_{S^{n-1}}\mathcal{A}(\rho,\theta)d\vol_{S^{n-1}}\right)\left(\int_{S^{n-1}}\int_0^\rho\mathcal{A}_H(r,\theta)dr\wedge d\vol_{S^{n-1}}\right)}{(\vol(B^H(\rho)))^2}\\
        &-\frac{\left(\int_{S^{n-1}}\mathcal{A}_H(\rho,\theta)d\vol_{S^{n-1}}\right)\left(\int_{S^{n-1}}\int_0^\rho\mathcal{A}(r,\theta)dr\wedge d\vol_{S^{n-1}}\right)}{(\vol(B^H(\rho)))^2}\\
        =&(\vol(B^H(\rho)))^{-2}\int_0^\rho\bigg(\left(\int_{S^{n-1}}\mathcal{A}(\rho,\theta)d\vol_{S^{n-1}}\right)\left(\int_{S^{n-1}}\mathcal{A}_H(r,\theta)d\vol_{S^{n-1}}\right)\\
        &-\left(\int_{S^{n-1}}\mathcal{A}_H(\rho,\theta)d\vol_{S^{n-1}}\right)\left(\int_{S^{n-1}}\mathcal{A}(r,\theta)d\vol_{S^{n-1}}\right)\bigg)dr.
    \end{align*}
    Therefore, to check
    \[\rho\mapsto\frac{\vol(B_p(\rho))}{\vol(B^H(\rho))}\]
    is nonincreasing, it's suffice to check
    \[r\mapsto\frac{\int_{S^{n-1}}\mathcal{A}(r,\theta)d\vol_{S^{n-1}}}{\int_{S^{n-1}}\mathcal{A}_H(r,\theta)d\vol_{S^{n-1}}}\]
    is nonincreasing.
    But we have
    \[\frac{\int_{S^{n-1}}\mathcal{A}(r,\theta)d\vol_{S^{n-1}}}{\int_{S^{n-1}}\mathcal{A}_H(r,\theta)d\vol_{S^{n-1}}}=\frac{1}{\omega_{n-1}}\int_{S^{n-1}}\frac{\mathcal{A}(r,\theta)}{\mathcal{A}_H(r,\theta)}d\vol_{S^{n-1}},\]
    which is nonincreasing by Theorem~\ref{thm:Ricci comparison}.
\end{proof}

\subsection{Comparison Inequality in Weak Sense}

Sometimes we will deal with functions with bad smoothness, so we need comparison theorems in weak sense.
Here we introduce a special case of Laplacian comparison inequality in distribution sense.

First, notice by Corollary~\ref{cor:Delta r=m}, we have
\begin{prop}
    Under same assumption of Theorem~\ref{thm:mean comparison}, mean curvature comparison inequality is equivalent to the following \emph{Laplacian comparison inequality}
    \[\Delta r\leq\Delta r_H.\]
\end{prop}

\begin{eg}
    In our seminar, we are concerned about $H=0$ case the most.
    We haven't computed $\Hess{r}$ for $\mathbb{R}^n$ yet.
    Let's compute for $p=0$ and $\Hess{r}(\rho)$.
    This is just the second fundamental form of sphere
    \[(x^1)^2+(x^2)^2+\cdots+(x^n)^2=\rho^2.\]
    Fix the ourward normal vector field $N=\frac{1}{\rho}\mathbf{x}$, then we have
    \[\II(X,Y)=\langle\nabla_XN,Y\rangle=\left\langle\frac{1}{\rho}X\mathbf{x},Y\right\rangle=\frac{1}{\rho}\langle X,Y\rangle.\]
    Hence we have
    \[\Hess{r}=\frac{1}{r}(g-dr\otimes dr).\]
    Take contraction we have
    \[\Delta r=\frac{n-1}{r}.\]

    We avoided using Jacobi field to compute Hessian of distance function.
    This method is limited, and is not recomended.
\end{eg}

\begin{thm}[Laplacian comparison]
    Let $(M,g)$ be a Riemannian manifold, with $\Ric\geq 0$.
    Then for the distance function to $p$, the inequality
    \[\Delta r\leq\frac{n-1}{r}\]
    holds in distribution sense, that is, for any $\varphi\in C^\infty_0(M)$, $\varphi\geq 0$, we have
    \[\int_Mr\Delta\varphi\leq\int_M\frac{n-1}{r}\cdot\varphi.\]
\end{thm}
\begin{proof}
    Let $M$ be decomposed into cut locus with respective to $p$ and a star-shaped domain, namely $M=:\Omega\sqcup\Cut(p)$.
    Lipschitz function $r$ is differentiable in $\Omega$, thus within $\Omega$ we have
    \[\Delta r\leq\frac{n-1}{r}.\]
    Fix $\varphi\in C^\infty_0(M)$, $\varphi\geq 0$.
    Since $|\Cut(p)|=0$, we have
    \[\int_Mr\Delta\varphi=\int_\Omega r\Delta\varphi.\]
    Since $\Omega$ is star-shaped, we can choose a increasing sequence $\Omega_k\subset\Omega$ such that
    \[\lim_{k\to\infty}\Omega_k=\Omega,\]
    and each $\Omega_k$ is obtained by shrinking $\Omega$ along $r$'s direction.
    Since Stokes' formula is valid for Lipschitz functions, and $\varphi$ has compact support, we have
    \begin{align*}
        \int_\Omega r\Delta\varphi&=-\int_\Omega\langle\nabla\varphi,\nabla r\rangle\\
        &=-\lim_{k\to\infty}\int_{\Omega_k}\langle\nabla\varphi,\nabla r\rangle.
    \end{align*}
    Last equality holds since $|\nabla r|=1$ within $\Omega$ and $\nabla\varphi$ is bounded, then apply Lebesgues' dominated convergence theorem.
    By Green's formula, we have
    \[-\int_{\Omega_k}\langle\nabla r,\nabla\varphi\rangle=\int_{\Omega_k}\Delta r\cdot\varphi-\int_{\partial\Omega_k}\varphi\cdot\frac{\partial r}{\partial\nu},\]
    where $\nu$ is the outer normal vector field.
    Since $\Omega_k$ is obtained by shrank $\Omega$ along $r$'s direction and $\varphi\geq 0$, we have
    \[\int_{\partial\Omega_k}\varphi\cdot\frac{\partial r}{\partial\nu}\geq 0.\]
    Thus we have
    \begin{align*}
        -\int_{\Omega_k}\langle\nabla r,\nabla\varphi\rangle&\leq\int_{\Omega_k}\Delta r\cdot\varphi\\
        &\leq\int_{\Omega_k}\frac{n-1}{r}\cdot\varphi.
    \end{align*}
    Finally we have
    \begin{align*}
        \int_Mr\Delta\varphi&\leq\lim_{k\to\infty}\int_{\Omega_k}\frac{n-1}{r}\cdot\varphi\\
        &=\int_{\Omega}\frac{n-1}{r}\cdot\varphi\\
        &=\int_M\frac{n-1}{r}\cdot\varphi,
    \end{align*}
    we used Lebesgues' dominated convergence theorem and $|\Cut(p)|=0$ again here.
\end{proof}

\section{Rigidity Results}

In this section, we present volume rigidity and diameter rigidity for closed and open manifolds respectively.

\subsection{Volume Rigidity}

Volume rigidity builds on the following rigidity result for volume comparison inequality.
\begin{prop}\label{prop:volume rigidity}
    Let $(M,g)$ be a Riemannian manifold with $\Ric\geq(n-1)H$.
    If there is an $R>0$ such that $\vol(B_p(R))=\vol(B^H(R))$, then $B_p(R)$ must be isometric to $B^H(R)$.
\end{prop}
\begin{proof}
    By volume comparison theorem, the function
    \[r\mapsto\frac{\vol(B_p(r))}{\vol(B^H(r))}\leq 1\]
    is nonincreasing, so if $\vol(B_p(R))=\vol(B^H(R))$ holds for some $R>0$, then $\vol(B_p(\rho))=\vol(B^H(\rho))$ holds for all $0<\rho<R$.
    Clearly we are before the cut locus.
    Let's trace back when the equality holds in volume comparison inequality.
    We used Cauchy--Schwarz inequality in
    \[|\Hess{r}|^2=|\II|^2\geq\frac{m^2}{n-1},\]
    hence if equality holds, $\II$ has equal eigenvalues.
    Thus $\partial B_p(\rho)$ is totally umbilical, in particular, $m(\rho,\theta)$ is constant on $\partial B_p(\rho)$.
    Under geodesic polar coordinate, with respective to the base point $p$, the metric $g$ has form
    \[g=dr^2+h(r,\theta).\]
    Since $\partial B_p(\rho)$ is totally umbilical, we have
    \[\II=\lambda(r)h\]
    for all $0<r<R$.
    Notice that
    \[\II_{ij}=\langle\nabla_i\partial_r,\partial_j\rangle=\partial_r\langle\partial_i,\partial_j\rangle-\langle\partial_i,\nabla_j\partial_r\rangle=\partial_rh_{ij}-\II_{ij},\]
    this implies $\II_{ij}=(1/2)\partial_{r}h_{ij}$.
    Thus we have an ordinary diffenrential equation
    \[\frac{1}{2}\partial_rh_{ij}=\lambda(r)h_{ij}.\]
    Solving this equation, we obtain
    \begin{equation}
        h_{ij}=(\varphi(r))^2h_{ij}(r_0,\theta).\label{eq:geodesic polar}
    \end{equation}
    We know the asymptotic expansion of Riemannian metric under geodesic polar coordinate is
    \[h_{ij}=r^2\sigma_{ij}+O(r^4),\]
    where $\sigma$ is the canonical metric of round sphere.
    Compare~\eqref{eq:geodesic polar}, we obtain $h_{ij}=f(r)\sigma_{ij}$, that is,
    \[g=dr^2+f(r)\sigma.\]
    Notice that $S^n(H)$ has metric
    \[g_H=dr\otimes dr+\sn^2_H(r)\sigma.\]
    Plug this into $\vol(B_p(\rho))=\vol(B^H(\rho))$, we have
    \[\int_0^\rho\int_{S^{n-1}}\sqrt{f(r)\det(g_{ij})}d\vol_{S^{n-1}}dr=\int_0^\rho\int_{S^{n-1}}\sqrt{\sn^2_H(r)\det(g_{ij})}d\vol_{S^{n-1}}dr.\]
    Notice that
    \[\int_{S^{n-1}}\sqrt{\det(g_{ij})}d\vol_{S^{n-1}}=\omega_{n-1},\]
    we thus obtain
    \[\int_0^\rho\sqrt{f(r)}dr=\int_0^\rho{\sn_H(r)}dr\]
    holds for all $0<\rho<R$.
    Therefore $f(r)=\sn^2_H(r)$ on $(0,R]$, and $B_p(R)$ is isometric to $B^H(R)$.
\end{proof}

Using Proposition~\ref{prop:volume rigidity}, we can derive volume rigidity for closed and open manifolds.

\begin{thm}[Volume rigidity for closed manifolds]
    Let $(M,g)$ be a Riemannian manifold with $\Ric\geq n-1$.
    If $\vol{M}=\vol{S^n}$, then $M$ is isometric to $S^n$.
\end{thm}
\begin{proof}
    By Bonnet--Myers' theorem, we have $\operatorname{diam}{M}\leq\pi$, hence
    \[\vol{(B_p(\pi))}=\vol{M}=\vol{S^n}=\vol{(B^1(\pi))}.\]
    Then apply Proposition~\ref{prop:volume rigidity}.
\end{proof}

\begin{thm}[Volume rigidity for open manifolds]
    Let $(M,g)$ be a Riemannian manifold with $\Ric\geq 0$.
    If
    \[\lim_{\rho\to\infty}\frac{n\vol{(B_p(\rho))}}{\omega_n\rho^n}=1,\]
    then $M$ is isometric to $\mathbb{R}^n$.
\end{thm}
\begin{proof}
    The function
    \[\rho\mapsto\frac{n\vol{(B_p(\rho))}}{\omega_n\rho^n}=\frac{\vol{(B_p(\rho))}}{\vol{(B^0(\rho))}}\]
    is nonincreasing and has limit $1$, hence
    \[\vol(B_p(\rho))\geq\vol(B^0(\rho)).\]
    By volume comparison inequality, the equality must hold, then apply Proposition~\ref{prop:volume rigidity}~to all $\rho>0$.
\end{proof}

\subsection{Diameter Rigidity}

Our result on diameter rigidity for closed manifolds is Shiu-Yuen Cheng's maximal diameter theorem.

\begin{thm}[Maximal diameter]
    Let $(M,g)$ be a Riemannian manifold with $\Ric\geq n-1$ and $\operatorname{diam}{M}=\pi$, then $M$ is isometric to $S^n$.
\end{thm}
\begin{proof}
    First notice that $M$ is compact by Bonnet--Myers' theorem, so we can choose $p,q\in M$ such that $|pq|=\pi$.
    Consider $B_p(\rho)$ and $B_q(\pi-\rho)$, by triangle inequality, we have $B_p(\rho)\cap B_q(\pi-\rho)=\varnothing$.
    Hence by volume comparison inequality, we have
    \begin{align*}
        \vol{M}&\geq\vol(B_p(\rho))+\vol(B_q(\pi-\rho))\\
        &=\frac{\vol(B_p(\rho))}{\vol(B^1(\rho))}\cdot\vol(B^1(\rho))+\frac{\vol(B_q(\pi-\rho))}{\vol(B^1(\pi-\rho))}\cdot\vol(B^1(\pi-\rho))\\
        &\geq\frac{\vol(B_p(\pi))}{\vol(B^1(\pi))}\cdot\vol(B^1(\rho))+\frac{\vol(B_q(\pi))}{\vol(B^1(\pi))}\cdot\vol(B^1(\pi-\rho))\\
        &=\frac{\vol{M}}{\vol{S^n}}\cdot(\vol(B^1(\rho))+\vol(B^1(\pi-\rho)))\\
        &=\frac{\vol{M}}{\vol{S^n}}\cdot\vol{S^n}\\
        &=\vol{M}.
    \end{align*}
    Therefore the equality of volume comparison inequality holds for $0<\rho\leq\pi$ on $M$, applying Proposition~\ref{prop:volume rigidity}~we obtain $B_p(\rho)$ is isometric to $B^1(\rho)$.
    In particular, $M=B_p(\pi)$ is isometric to $B^1(\pi)=S^n$.
\end{proof}

We need to explain what does ``diameter rigidity'' mean for an open manifold.
Being of infinite diameter is not enough, there should be two points at infinity.
That is, there should be a geodesic adjoining two points at infinity, namely, a line exists.

\begin{defn}
    Let $\gamma:\mathbb{R}\to M$ be a geodesic, if $\gamma$ is a global isometry, i.\ e.\ for any $s,t\in\mathbb{R}$ the equality $|\gamma(s)\gamma(t)|=|s-t|$ holds, then $\gamma$ is called a \emph{line}.
\end{defn}

We expect rigidity result happens when a line exists.
This leads to Cheeger--Gromoll's splitting theorem.

\begin{thm}[Splitting]
    Let $(M,g)$ be a simply-connected Riemannian manifold with $\Ric\geq 0$, and a line exists on $M$.
    Then $M$ is isometric to $\mathbb{R}\times N$ with product metric.
\end{thm}

Before proving this theorem, we need to input several big theorems.

\begin{thm}[Strong maximum principle]
    Let $(M,g)$ be a Riemannian manifold, $f$ be a Lipschitz function on $M$.
    If $f$ satisfies $\Delta f\leq 0$ in the distribution sense and has a global minimum, then $f$ is constant on $M$.
\end{thm}

\begin{thm}[Weyl's lemma]
    Let $(M,g)$ be a Riemannian manifold, $f$ be a Lipschitz function.
    If $\Delta f=0$ in the distribution sense, then $f$ is smooth.
\end{thm}

\begin{thm}[De Rham decomposition]
    Let $(M,g)$ be a simply-connected Riemannian manifold.
    Let $\mathcal{F}$ be a distribution that is parallel with respective to the Levi--Civita connection, $\mathcal{F}^\perp$ be its orthogonal complement.
    For any $p\in M$, let $N,N^\perp$ be the maximal integral submanifold of $\mathcal{F},\mathcal{F}^\perp$ at $p$.
    Equip $N,N^\perp$ with induced metric from $M$, then $M$ is isometric to $N\times N^\perp$ with product metric.
\end{thm}

We will not prove these three theorems here.

\begin{proof}[Proof of splitting theorem]
    The proof uses \emph{Busemann function}.
    We first define and discuss some properties of Busemann function.

    Fix $p\in M$, let $\gamma$ be a line on $M$.
    For $t\in\mathbb{R}$, define $b^+_t(p)=|p\gamma(t)|-t$.
    Then by triangle inequality, if $t_1>t_2$, we have
    \[b^+_{t_1}(p)-b^+_{t_2}(p)=|p\gamma(t_1)|-|p\gamma(t_2)|-(t_1-t_2)\leq 0,\]
    hence $b^+_t(p)$ is nonincreasing with respective to $t$.
    Moreover, by triangle inequality, we have $b^+_t(p)\geq-|p\gamma(0)|$, hence $b^+_t(p)$ is bounded below.
    Thus we can take limit
    \[B^+(p):=\lim_{t\to+\infty}b^+_t(p).\]
    Similarly we define $b^-_t(p)=|p\gamma(-t)|-t$, we have the limit
    \[B^-(p):=\lim_{t\to+\infty}b^-_t(p)\]
    is well-defined.

    Now we discuss the properties of Busemann functions.
    First we notice that
    \[B^+(p)+B^-(p)=|p\gamma(t)|+|p\gamma(-t)|-2t\geq 0.\]
    Then by Laplacian comparison inequality, we have for any test function $\varphi\in C^\infty_0(M)$, $\varphi\geq 0$
    \[\int_{M}b^+_t\Delta\varphi\leq\int_M\frac{n-1}{b^+_t+t}\cdot\varphi.\]
    Let $t\to+\infty$, we obtain (we may use Lebesgue's dominated convergence theorem)
    \[\int_MB^+\Delta\varphi\leq 0.\]
    Similarly we have
    \[\int_MB^-\Delta\varphi\leq 0.\]
    Thus $\Delta(B^++B^-)\leq 0$ in the distribution sense.
    Moreover, clearly $B^++B^-=0$ on $\gamma$, hence $B^++B^-$ satisfies the assumption of strong maximum principle, we have $B^++B^-=0$.
    In particular, we have $\Delta(B^++B^-)=0$ in the classical sense, hence $\Delta B^+=\Delta B^-=0$ in the distribution sense.
    By Weyl's lemma, we have $B^+$ and $B^-$ are both smooth.
    Moreover, since the gradient of distance function has modulus $1$ almost everywhere, we have $|\nabla B^+|=1$.
    Using Bochner's formular, we have
    \[0=|\Hess{B^+}|^2+\langle\nabla B^+,\nabla\Delta B^+\rangle+\Ric(\nabla B^+,\nabla B^+)\geq|\Hess{B^+}|^2,\]
    hence $\Hess{B^+}=0$.

    The rest of part is easy geometry.
    Since covariant differentiation is commutative with musical isomorphism, we have $\nabla(\nabla B^+)=(\Hess{B^+})^\sharp=0$.
    Hence $\nabla B^+$ is parallel.
    Choose a regular point $p$ of $B^+$, by de Rham's decomposition theorem, $M$ is isometric to the product of integral curve of $\nabla B^+$ through $p$ and the level set of $B^+$ at $p$.
    Without loss of generality we can assume $p$ lies on $\gamma$, then since $\nabla B^+$ is parallel, the integral curve is the geodesic through $p$, i.\ e.\ the line.
\end{proof}

\end{document}
\section{Tensors}

\subsection{Notation}

We first fix the notation of tensors.
Let $(M,g)$ be a Riemannian manifold.
A \emph{tensor} $T$ of type $(r,s)$ is a smooth section of vector bundle
\[T^{(r,s)}(TM)=\left(\bigotimes_{i=1}^rTM\right)\otimes\left(\bigotimes_{j=1}^sT^*M\right).\]
On each fiber, a tensor can be regarded as a multilinear map
\[T|_{p}:\underbrace{T^*_pM\times\cdots\times T^*_pM}_{r}\times\underbrace{T_pM\times\cdots\times T_pM}_{s}\to\mathbb{R},\]
so we can talk about symmetric and positive-definite tensor (for $(0,2)$-tensor), alternating tensor (for $(0,n)$-tensor) well.
Tensors of type $(r,0)$ are called contravariant tensors, and of type $(0,s)$ are called covariant tensors.

Let $(x^1,\cdots,x^n)$ be a local coordinate.
We will adopt the notation $\{\partial_1,\cdots,\partial_n\}$ for a local frame for $TM$, and $\{dx^1,\cdots,dx^n\}$ for a local frame for $T^*M$.
We will adopt the \textsc{Einstein} summation convention, so the local expression for an $(r,s)$-tensor $T$ is
\[T=T^{i_1\cdots i_r}_{j_1\cdots j_s}\partial_{i_1}\otimes\cdots\otimes\partial_{i_r}\otimes dx^{j_1}\otimes\cdots\otimes dx^{j_s}.\]

The tangent space $TM$ and cotangent space $T^*M$ are canonically isomorphic via \emph{musical isomorphism}.
Under a local coordinate $(x^1,\cdots,x^n)$, let $g=g_{ij}$.
Then we have the musical isomorphism (``lowering index'')
\begin{align*}
    \flat:TM&\to T^*M\\
    X^i\partial_i&\mapsto g_{jk}X^kdx^j,
\end{align*}
and we denote $X=X^i\partial_i$, $X^\flat=g_{jk}X^kdx^j$.
The inverse is given by (``raising index'')
\begin{align*}
    \sharp:T^*M&\to TM\\
    \omega_idx^i&\mapsto g^{jk}\omega_k\partial_j,
\end{align*}
and we denote $\omega=\omega_idx^i$, $\omega^\sharp=g^{jk}\omega_k\partial_j$.
Clearly musical isomorphism can be extended to arbitrary tensors.

\subsection{Contration}

We discuss contraction of two indices of a tensor in this subsection.

First, let us check this naive example.
Let $V$ be an $n$-dimensional Euclidean space with flat metric (i.e.\ with metric $\delta_{ij}$), $S:V\to V$ be a (symmetric) linear transformation, $L$ be its associated bilinear function.
Let $S$ and $L$ has matrices
\[\begin{bmatrix}
    a^1_1 & a^1_2 & \cdots & a^1_n \\
    a^2_1 & a^2_2 & \cdots & a^2_n \\
    \vdots & \vdots & \ddots & \vdots \\
    a^n_1 & a^n_2 & \cdots & a^n_n
\end{bmatrix}\quad\text{and}\quad\begin{bmatrix}
    a_{11} & a_{12} & \cdots & a_{1n} \\
    a_{21} & a_{22} & \cdots & a_{2n} \\
    \vdots & \vdots & \ddots & \vdots \\
    a_{n1} & a_{n2} & \cdots & a_{nn}
\end{bmatrix},\]
the matrices of $S$ and $L$ are related by the musical isomorphism of Euclidean metric $\delta_{ij}$, since we know $a^i_j=a_{ij}$.
Clearly we want their trace or contraction to be the same.
To define the contraction of $S$ is relatively easy: $S$ has expression
\[S=a^i_jv^j\otimes v_i^*,\]
where we take $\{v^i\},\{v_i^*\}$ to be a basis and whose dual basis of $V$ respectively.
Plug $v^j$ into $v_i^*$ and take summation, we obtain
\[\tr_{1,2}S=\sum_{i=1}^na^i_i.\]
Since we want
\[\tr_{1,2}L=\sum_{i=1}^na_{ii},\]
this enlighten us that two tensors should have same contraction modulo a musical isomorphism.

Thus we have the following definition.
\begin{defn}
    Let $(M,g)$ be a Riemannian manifold, $(x^1,\cdots,x^n)$ be a local coordinate.
    Let $S=S^i_j\partial_i\otimes dx^j$ be an $(1,1)$-tensor, then the \emph{contraction} of indices $1,2$ is defined to be
    \[\tr_{1,2}S=S^i_i.\]
    Let $L=L_{ij}dx^i\otimes dx^j$ be an $(0,2)$-tensor, then the contraction of indices $1,2$ is defined to be
    \[\tr_{1,2}L=g^{ij}L_{ij}.\]
    Similarly, we can define the contraction of tensors of type $(r,s)$.
\end{defn}

\subsection{Norm of a Tensor}

We still look at our very first definition of norm.
We use an exaggerated way to write the norm of a vector field $X$:
\begin{align*}
    |X|^2&=g(X,X)\\
    &=\tr_{1,2}\left(\tr_{1,3}(g\otimes X)\otimes X\right)\\
    &=\tr_{1,2}(X^\flat\otimes X).
\end{align*}
If we think $X:T^*M\to\mathbb{R}$ is a function on $T^*M$, then $X^\flat:TM\to\mathbb{R}$ is its adjoint, which can be denoted by $X^*$.
Thus we reached the definition.

\begin{defn}
    Let $(M,g)$ be a Riemannian manifold, and $T$ be an $(r,s)$-tensor on $M$.
    Let $T^*$ be the $(s,r)$-tensor related to $T$ with musical isomorphism, we permute $T^*$ with covariant part lying before contravariant part.
    Define the \emph{norm} of $T$ by
    \[|T|^2=\tr_{1,r+s+1}\tr_{2,r+s+2}\cdots\tr_{r+s,2r+2s}(T^*\otimes T).\]
\end{defn}

\begin{eg}
    We will use the norm of a Hessian, that is, a $(0,2)$-tensor.
    Let $L=L_{ij}dx^i\otimes dx^j$ be such a tensor.
    Then we have
    \[L^*=g^{ik}g^{jl}L_{kl}\partial_i\otimes\partial_j,\]
    and
    \[L^*\otimes L=g^{im}g^{jn}L_{mn}L_{kl}\partial_i\otimes\partial_j\otimes dx^k\otimes dx^l.\]
    We take contraction and obtain
    \[\tr_{1,3}\tr_{2,4}(L^*\otimes L)=g^{ik}g^{jl}L_{ij}L_{kl}.\]
\end{eg}

\subsection{Covariant Derivative and Covariant Differentiation}

Let $(M,g)$ be a Riemannian manifold and $\nabla$ be its Levi--Civita connection.
Recall that given a vector field $Y$ and a tangent vector $X_p\in T_pM$ (let it be the restriction of $X$ at $p$), the covariant derivative $\nabla_XY(p)$ can be evaluated as follows:
Choose an arbitrary curve $\gamma:I\to M$ with $\gamma(0)=p,\dot\gamma(0)=X_p$, let $P_t$ be the parallel transportation of $\nabla$ along $\gamma$.
Then we have
\[\nabla_XY|_p=\lim_{t\to 0}\frac{1}{t}\left(P^{-1}_t(Y(\gamma(t))-Y(\gamma(0)))\right).\]

Generalize this idea to tensors, just as we did for Lie derivative, we can reach the definition of covariant derivative of tensors.
\begin{defn}
    Let $T$ be an $(r,s)$-tensor, $p\in M$ and $X_p\in T_pM$ (let it be the restriction of $X$ at $p$).
    We define $\nabla_XT(p)$ as follows:
    Choose an arbitrary curve $\gamma:I\to M$ with $\gamma(0)=p,\dot\gamma(0)=X_p$, let $P_t$ be the parallel transportation of $\nabla$ along $\gamma$.
    Define $P^\otimes_t:T^{(r,s)}T_{p}M\to T^{(r,s)}T_{\gamma(t)}M$ by
    \[P^\otimes_t=\underbrace{P_t\otimes\cdots\otimes P_t}_{r}\otimes\underbrace{(P^*_t)^{-1}\otimes\cdots\otimes(P^*_t)^{-1}}_{s},\]
    and then we define
    \[\nabla_XT|_p=\lim_{t\to 0}\frac{1}{t}\left((P^\otimes_t)^{-1}(T(\gamma(t)))-T(\gamma(0))\right).\]
\end{defn}

We also have the notion of covariant differentiation.
\begin{defn}
    Let $T$ be an $(r,s)$-tensor.
    The \emph{covariant differentiation} $\nabla T$ of $T$ is an $(r,s+1)$-tensor such that
    \[\nabla T(\cdots,X)=\nabla_XT(\cdots).\]
\end{defn}

\begin{eg}
    The metric compatibility of Levi--Civita connection is equivalent to $\nabla g=0$.
    It's hard to verify this property by now, but we will soon figure out how to compute covariant derivative.
\end{eg}

Two properties are essential to compute the covariant derivative.
We write a lemma for this.

\begin{lem}
    \begin{enumerate}[\rm(1)]
        \item Covariant derivative satisfies the \emph{Leibniz law}, that is, for tensors $S,T$, we have
        \[\nabla_X(S\otimes T)=S\otimes(\nabla_XT)+(\nabla_XS)\otimes T.\]
        \item Covariant derivative commutes with contraction, that is, for tensor $T$, we have
        \[\nabla_X(\tr_{i,j}T)=\tr_{i,j}\nabla_XT.\]
    \end{enumerate}
\end{lem}
\begin{proof}
    For simplicity, we prove for tensor $X\otimes\omega$, a $(1,1)$-tensor.
    General case is similar.
    Choose a curve $\gamma:I\to M$, $\dot\gamma(0)=v$, and a parallel basis $\{e_i(t)\}$ along $\gamma$.
    Let $\{\alpha^i(t)\}$ be the dual basis with respective to $\{e^i(t)\}$, and let
    \begin{align*}
        X(\gamma(t))&=X^i(t)e_i(t),\\
        \omega(\gamma(t))&=\omega_i(t)\alpha^i(t).
    \end{align*}
    Thus we have
    \begin{align*}
        \nabla_v(X\otimes\omega)&=\left.\frac{d}{dt}\right|_{t=0}(X^i(t)\omega_j(t))e_i(0)\otimes\alpha^j(0)\\
        &=\left(\dot{X}^i(0)\omega_j(0)+X^i(0)\dot\omega_j(0)\right)e_i(0)\otimes\alpha^j(0)\\
        &=(\nabla_vX)\otimes\omega+X\otimes(\nabla_v\omega).
    \end{align*}
    Moreover, we have
    \begin{align*}
        \nabla_v(\tr_{1,2}X\otimes\omega)&=\left.\frac{d}{dt}\right|_{t=0}(X^i(t)\omega_i(t))\\
        &=\dot{X}^i(0)\omega_i(0)+X^i(0)\dot\omega_i(0)\\
        &=\tr_{1,2}(\nabla_vX)\otimes\omega+\tr_{1,2}X\otimes(\nabla_v\omega)\\
        &=\tr_{1,2}(\nabla_v(X\otimes\omega)).
    \end{align*}
    Thus we proved the lemma.
\end{proof}

\begin{eg}
    In this example, we illustrate how to calculate the covariant derivative of a covariant tensor.
    Let $T$ be a $(0,s)$-tensor, we want to know what is
    \[(\nabla_XT)(X_1,\cdots,X_s).\]
    For simplicity we let $s=2$, there is no difference for general case.
    As we did before, we write $T(X_1,X_2)$ into a form of contraction, and then use the commutativity of contraction and covariant derivative.
    It writes
    \begin{align*}
        XT(X_1,X_2)=&\nabla_X(T(X_1,X_2))\\
        =&\nabla_X(\tr_{1,3}\tr_{2,4}T\otimes X_1\otimes X_2)\\
        =&\tr_{1,3}\tr_{2,4}\nabla_X(T\otimes X_1\otimes X_2)\\
        =&\tr_{1,3}\tr_{2,4}(T\otimes(\nabla_XX_1)\otimes X_2+T\otimes X_1\otimes(\nabla_XX_2))\\
        &+\tr_{1,3}\tr_{2,4}((\nabla_XT)\otimes X_1\otimes X_2)\\
        =&(\nabla_XT)(X_1,X_2)+T(\nabla_XX_1,X_2)+T(X_1,\nabla_XX_2),
    \end{align*}
    thus we have
    \begin{equation}
        (\nabla_XT)(X_1,X_2)=XT(X_1,X_2)-T(\nabla_XX_1,X_2)-T(X_1,\nabla_XX_2).\label{eq:covdiff eq1}
    \end{equation}
    In particular, if we take $T=g$, then the equation~\eqref{eq:covdiff eq1}~is nothing but the metric compatibility of Levi--Civita connection.
\end{eg}

\begin{eg}
    Sometimes calculating covariant derivative in a local chart is useful.
    Let $\nabla dx^i=\omega_{jk}dx^j\otimes d^k$, then we have
    \begin{align*}
        \omega_{jk}&=(\nabla dx^i)(\partial_j,\partial_k)\\
        &=(\nabla_kdx^i)(\partial_j)\\
        &=\partial_k(dx^i(\partial_j))-dx^i(\nabla_k\partial_j)\\
        &=-\Gamma_{kj}^i.
    \end{align*}
    Thus we have $\nabla dx^i=-\Gamma_{kj}^idx^j\otimes dx^k$.
\end{eg}

\subsection{Curvature Endomorphism}

We next consider second covariant differentiation.
We first introduce a symbol.

\begin{defn}
    Let $T$ be a tensor, $X,Y$ be vector fields, we use $\nabla_{X,Y}^2T$ to denote the tensor
    \[\nabla^2_{X,Y}(\cdots):=\nabla(\nabla T)(\cdots,Y,X).\]
\end{defn}

We have an explicit formula for second covariant differentiation.

\begin{lem}\label{lem:second covdiff}
    We have
    \[\nabla_{X,Y}^2T=\nabla_X\nabla_YT-\nabla_{\nabla_XY}T.\]
\end{lem}

We leave this lemma as an exercise.
(Just remember how do you calculate Hessian in Riemannian geometry class.)

\begin{defn}
    The \emph{curvature endomorphism} is given by
    \begin{align*}
        R(X,Y):\Gamma\left(T^{(r,s)}TM\right)&\to\Gamma\left(T^{(r,s)}TM\right)\\
        T&\mapsto\nabla^2_{Y,X}T-\nabla^2_{X,Y}T.
    \end{align*}
\end{defn}

\begin{prop}[Ricci identity]
    We have
    \[R(X,Y)T=-\nabla_X\nabla_YT+\nabla_Y\nabla_XT+\nabla_{[X,Y]}T.\]
\end{prop}

\begin{rem}
    There are many ways to interpret curvature.
    One way (maybe the most common way) to understand curvature is that curvature measures the deviation of map $X\mapsto\nabla_X$ from being a Lie algebra homomorphism, as the Ricci identity indicates.
    Another way is that curvature measures the deviation of second covariant derivative from being commutative.
    However, this viewpoint cannot explain the existence of $\nabla_{[X,Y]}$ term.
    We choose here the viewpoint of curvature measures the deviation of second covariant differentiation being commutative.

    Last but not least, beware of our sign convention.
\end{rem}